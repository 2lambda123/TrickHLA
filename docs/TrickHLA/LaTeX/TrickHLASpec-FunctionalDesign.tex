%%%%%%%%%%%%%%%%%%%%%%%%%%%%%%%%%%%%%%%%%%%%%%%%%%%%%%%%%%%%%%%%%%%%%%%%%
%
% File: TrickHLASpec-FunctionalDesign.tex
%
% Purpose: TrickHLA Product Specification -- Functional Design chapter
%
%%%%%%%%%%%%%%%%%%%%%%%%%%%%%%%%%%%%%%%%%%%%%%%%%%%%%%%%%%%%%%%%%%%%%%%%%
\chapter{Functional Design}\label{sec:functional_design}

This chapter presents the detailed design of the \TrickHLA\ model.
We present the model in terms of its C++ classes,
grouping them into low, middle, and high-level classes.
We proceed from the bottom up.

%%%%%%%%%%%%%%%%%%%%%%%%%%%%%%%%%%%%%%%%%%%%%%%%%%%%%%%%%%%%%%%%%%%%%%%%%
\section{Low Level Classes and Types}

These low level classes are not particularly interesting.
The are mainly miscellaneous utilities and low level data types
used in other parts of the system.

% -----------------------------------------------------------------------
\subsection{{\tt TrickHLA::Utilities}}

\paragraph{Description.}
This class contains miscellaneous utility methods.
In the current version, these are all static byteswapping routines.

\paragraph{Files.}
The class files are shown below.
   
{
  \scriptsize
  \begin{tabular}{|l|l|} 
    \hline
    file name & description \\
    \hline \hline
    {\tt TrickHLA/Utilities.hh} 
    & standard C++ header file
    \\ \hline
    {\tt TrickHLA/Utilities.cpp} 
    & implementation of the byteswap methods
    \\ \hline
  \end{tabular}
}

% -----------------------------------------------------------------------
\subsection{{\tt TrickHLA::StringUtilities}}

\paragraph{Description.}
A set of static string utilities.
In the current version, these convert between C, C++ and wide
string representations.

\paragraph{Files.}
The class files are shown below.
   
{
  \scriptsize
  \begin{tabular}{|l|l|} 
    \hline
    file name & description \\
    \hline \hline
    {\tt TrickHLA/StringUtilities.hh} 
    & standard C++ header file
      (all the utilities are defined inline)
    \\ \hline
  \end{tabular}
}

% -----------------------------------------------------------------------
\subsection{{\tt TrickHLA::KnownFederate}}

\paragraph{Description.}
A class that holds information about the known federates in a federation.
This class is used by the \TrickHLA\ model to wait for all known federates
before starting the simulation.
The class is really meant as a structure. Its data fields are public.
The fields specify the known federate's {\em name} and whether or not
the federate is required before the simulation starts.

\paragraph{Files.}
The class files are shown below.
   
{
  \scriptsize
  \begin{tabular}{|l|l|} 
    \hline
    file name & description \\
    \hline \hline
    {\tt TrickHLA/KnownFederate.hh} 
    & standard C++ header file
    \\ \hline
  \end{tabular}
}

% -----------------------------------------------------------------------
\subsection{{\tt TrickHLA::Int64Interval}}

\paragraph{Description.}
An implementation of HLA time intervals based on a 64 bit integer that stores
microseconds.
It is a subclass of the HLA class, {\tt LogicalTimeInterval}.
It implements all the methods required by that class.

\paragraph{Files.}
The class files are shown below.
   
{
  \scriptsize
  \begin{tabular}{|l|l|} 
    \hline
    file name & description \\
    \hline \hline
    {\tt TrickHLA/Int64Interval.hh} 
    & standard C++ header file
    \\ \hline
    {\tt TrickHLA/Int64Interval.cpp} 
    & method implementation file
    \\ \hline
  \end{tabular}
}

% -----------------------------------------------------------------------
\subsection{{\tt TrickHLA::Int64Time}}
\paragraph{Description.}
An implementation of HLA time based on a 64 bit integer that stores
microseconds.
It is a subclass of the HLA class, {\tt LogicalTime}.
It implements all the methods required by that class.

\paragraph{Files.}
The class files are shown below.
   
{
  \scriptsize
  \begin{tabular}{|l|l|} 
    \hline
    file name & description \\
    \hline \hline
    {\tt TrickHLA/Int64Time.hh} 
    & standard C++ header file
    \\ \hline
    {\tt TrickHLA/Int64Time.cpp} 
    & method imlemenation file
    \\ \hline
  \end{tabular}
}

% -----------------------------------------------------------------------
\subsection{{\tt TrickHLA::Types}}

\paragraph{Description.}
enums and typedefs used by various parts of the system

\paragraph{Files.}
The class files are shown below.
   
{
  \scriptsize
  \begin{tabular}{|l|l|} 
    \hline
    file name & description \\
    \hline \hline
    {\tt TrickHLA/Types.hh} 
    & enum definitions
    \\ \hline
  \end{tabular}
}

% -----------------------------------------------------------------------
\subsection{{\tt TrickHLA::DebugHandler}}

\paragraph{Description.}
A class which defines the verbosity level and the code sections which are to emit
messages. Unless overridden in the input file, no messsages are emitted by the
\TrickHLA\ software -- only the simulation messages, if any, will get emitted.

\paragraph{Files.}
The class files are shown below.
   
{
  \scriptsize
  \begin{tabular}{|l|l|} 
    \hline
    file name & description \\
    \hline \hline
    {\tt TrickHLA/DebugHandler.hh} 
    & standard C++ header file
    \\ \hline
  \end{tabular}
}


%%%%%%%%%%%%%%%%%%%%%%%%%%%%%%%%%%%%%%%%%%%%%%%%%%%%%%%%%%%%%%%%%%%%%%%%%
\section{Middle Level Classes}

The middle level classes represent core HLA abstractions:
synchronization points,
objects, attributes, interactions, and parameters.

% -----------------------------------------------------------------------
\subsection{{\tt TrickHLA::SyncPntListBase}}

\paragraph{Description.}
This class provides a mechanism for storing and managing HLA
synchronization points.
A set of public methods is exported that enable synchronization points
to be added to the system, achieved, acknowledged, and queried.

\paragraph{Files.}
The class files are shown below.
   
{
  \scriptsize
  \begin{tabular}{|l|l|} 
    \hline
    file name & description \\
    \hline \hline
    {\tt TrickHLA/SyncPntListBase.hh} 
    & standard C++ header file
    \\ \hline
    {\tt TrickHLA/SyncPntListBase.cpp} 
    & method implementation file
    \\ \hline
  \end{tabular}
}

% -----------------------------------------------------------------------
\subsection{{\tt TrickHLA::Object}}

\paragraph{Description.}
This class represents HLA objects.
It is large class with many methods. 
The higher level classes delegate many tasks to this class.
The public methods associate the object with the high-level
{\tt TrickHLA::Manager} class, handle name reservation, publishing,
subscribing, object registration, instance removal,
publishing and subscribing of Trick variables,
ownership transfer, object deletion, and lag compensation.
Lag compensation and ownership transfer are delegated to associated
helper classes.
It also exports method related to its constituent attributes which are
referenced through a private array of {\tt TrichHLA::Attribute} instances.

\paragraph{Files.}
The class files are shown below.
   
{
  \scriptsize
  \begin{tabular}{|l|l|} 
    \hline
    file name & description \\
    \hline \hline
    {\tt TrickHLA/Object.hh} 
    & standard C++ header file
    \\ \hline
    {\tt TrickHLA/Object.cpp} 
    & method implemetation file
    \\ \hline
  \end{tabular}
}

% -----------------------------------------------------------------------
\subsection{{\tt TrickHLA::Attribute}}

\paragraph{Description.}
This class represents an HLA attribute.

\paragraph{Files.}
The class files are shown below.
   
{
  \scriptsize
  \begin{tabular}{|l|l|} 
    \hline
    file name & description \\
    \hline \hline
    {\tt TrickHLA/Attribute.hh} 
    & standard C++ header file
    \\ \hline
    {\tt TrickHLA/AttributeNoICG.hh} 
    & C++ header file that is not preprocessed by Trick
    \\ \hline
    {\tt TrickHLA/Attribute.cpp} 
    & method implementation file
    \\ \hline
  \end{tabular}
}

% -----------------------------------------------------------------------
\subsection{{\tt TrickHLA::Interaction}}

\paragraph{Description.}
This class represents an HLA interaction.
Its methods are responsible for sending interactions to remote federates
and receiving remotely generated interations. 
The logic for handling incoming interactions is delegated to a
helper class.

\paragraph{Files.}
The class files are shown below.

{
  \scriptsize
  \begin{tabular}{|l|l|} 
    \hline
    file name & description \\
    \hline \hline
    {\tt TrickHLA/Interaction.hh} 
    & standard C++ header file
    \\ \hline
    {\tt TrickHLA/Interaction.cpp} 
    & method implemetation file
    \\ \hline
  \end{tabular}
}

% -----------------------------------------------------------------------
\subsection{{\tt TrickHLA::Parameter}}

\paragraph{Description.}
This class represents HLA parameters of an interaction.
The public methods support manipulation of the parameter's
FOM name, HLA handle, and encoded and decoded values.

\paragraph{Files.}
The class files are shown below.
   
{
  \scriptsize
  \begin{tabular}{|l|l|} 
    \hline
    file name & description \\
    \hline \hline
    {\tt TrickHLA/Parameter.hh} 
    & standard C++ header file
    \\ \hline
    {\tt TrickHLA/Parameter.cpp} 
    & method imlementation file
    \\ \hline
  \end{tabular}
}

%%%%%%%%%%%%%%%%%%%%%%%%%%%%%%%%%%%%%%%%%%%%%%%%%%%%%%%%%%%%%%%%%%%%%%%%%
\section{Upper Level Classes}

The high level classes represent \TrickHLA\ functionality.
Some of these classes must be subclasses by simulation developers
and others are not intended for direct use by developers at all but are
rather invoked as Trick jobs from the default \TrickHLA\ 
{\ttfamily sim\_object}.

% -----------------------------------------------------------------------
\subsection{{\tt TrickHLA::Packing}}

\paragraph{Description.}
This class is a base class from which simulation developers may
incorporate their own pack/unpack functionality into the \TrickHLA\ model.
The virtual methods {\tt pack()} and {\tt unpack()} have default implementations
that will terminate the simulation, therefore you must subclass the base class
in order to use it, overriding the two methods

The \TrickHLA\ infrastructure invokes the {\tt pack()} method prior to
sending data to remote federates, and it invokes {\tt unpack()} upon receiving
data from remote federates.

\paragraph{Files.}
The class files are shown below.
   
{
  \scriptsize
  \begin{tabular}{|l|l|} 
    \hline
    file name & description \\
    \hline \hline
    {\tt TrickHLA/Packing.hh} 
    & standard C++ header file
    \\ \hline
    {\tt TrickHLA/Packing.cpp} 
    & method imlementation file
    \\ \hline
  \end{tabular}
}

% -----------------------------------------------------------------------
\subsection{{\tt TrickHLA::OwnershipHandler}}

\paragraph{Description.}
This class handles HLA attribute ownerhip transfer.
It does so by queueing up transfer requests. 
The queued requests are processed by a job scheduled in the
default \TrickHLA\ {\tt sim\_object}.
Ownership may be divested using the {\tt push\_ownership()} method.
It may be acquired by useing the {\tt pull\_ownership()} method.
The actual work of pushing/pulling ownership is delegated to
ownership-related methods in the associated {\tt TrickHLA::Object} instance.

\paragraph{Files.}
The class files are shown below.
   
{
  \scriptsize
  \begin{tabular}{|l|l|} 
    \hline
    file name & description \\
    \hline \hline
    {\tt TrickHLA/OwnershipHandler.hh} 
    & standard C++ header file
    \\ \hline
    {\tt TrickHLA/OwnershipHandler.cpp} 
    & method implementation file
    \\ \hline
  \end{tabular}
}

% -----------------------------------------------------------------------
\subsection{{\tt TrickHLA::LagCompensation}}

\paragraph{Description.}
This is a base class that allows simulation developers to insert logic
that compensates for HLA-induced lags when attribute ownership is transfered
from one federate to another.
The two methods,
{\tt send\_lag\_compensation()} and {\tt receive\_lag\_compensation()}
are mutually exclusive.
An object is configured to have send-side, receive-side, or no lag compensation.
A simulation developer must subclass this base class, create an instance
of it in the simulation {\tt S\_define} file, associate that instance
with a particular object, and indicate whether the compensation is send-side
or receiving-side compensation for the \TrickHLA\ compensation logic to work.
This compensation logic is invoked jobs in the default \TrickHLA\ 
{\tt sim\_object}. These jobs defer to the {\tt TrickHLA::Object} class 
which in turn invokes application-specific lag compensation methods (if any).

\paragraph{Files.}
The class files are shown below.
   
{
  \scriptsize
  \begin{tabular}{|l|l|} 
    \hline
    file name & description \\
    \hline \hline
    {\tt TrickHLA/LagCompensation.hh} 
    & standard C++ header file
    \\ \hline
    {\tt TrickHLA/LagCompensation.cpp} 
    & method implementation file
    \\ \hline
  \end{tabular}
}

% -----------------------------------------------------------------------
\subsection{{\tt TrickHLA::InteractionHandler}}

\paragraph{Description.}
This class handles the sending and receiving of HLA insteractions.

Incoming interactions are dispatched from the federate ambassador
to the manager class (see below) which in turn delegates responsibility
to a corresponding instance of the {\tt TrickHLA::Interaction} class
which finally delegates to the associted application-specific
interaction handler.
Application-specific handlers must subclass this class.

This class implements several methods for sending interactions. 
These may be called from simulation code, or they may be scheduled as
Trick jobs in the {\tt S\_define} file.

\paragraph{Files.}
The class files are shown below.
   
{
  \scriptsize
  \begin{tabular}{|l|l|} 
    \hline
    file name & description \\
    \hline \hline
    {\tt TrickHLA/InteractionHandler.hh} 
    & standard C++ header file
    \\ \hline
    {\tt TrickHLA/InteractionHandler.cpp} 
    & method implementation file
    \\ \hline
  \end{tabular}
}

% -----------------------------------------------------------------------
\subsection{{\tt IMSim::FreezeInteractionHandler}}

\paragraph{Description.}
NOTE: This section needs to be removed. This is now part of the DSES and
IMSim {\ttfamily ExecutionControl} classes.

This class handles the sending and receiving of Freeze Interactions in
TimeStamp Order. It is available only when using version 2 of the multiphase
initialization (by specifiying "{\tt THLA.manager.sim\_initialization\_scheme = THLA\_MULTIPHASE\_INIT\_V2;}"
in the input file).

It is a subclass of {\tt TrickHLAInteractionHandler}, specialized to 
send and receive an interaction that will suspend the federation execution. It 
requires that the {\tt THLA.federate.check\_freeze\_time()} routine is present 
in the {\tt S\_define} file (see Figure~\ref{fig:default-sim-object}).

When the user wishes to send a {\tt Freeze Interaction}, this class will check 
if the supplied time is valid. If the supplied time is a valid future time, 
this class sends out an interaction for this future time.

If no time was supplied, the next available cooridinated {\tt freeze} time is 
computed and sent out as the {\tt interaction time}. Note: This time 
computation takes into account interaction times sent by late-joining federates.

All federates will go into {\tt freeze} mode at the bottom of the 
frame specified by the interaction time.

The user is responsible for {\tt un-freezing} the sim after the {\tt freeze} 
mode work has completed.

This class implements methods for sending and receiving of these specialized 
interactions. These may be called from simulation code, or they may be 
scheduled as Trick jobs in the {\tt S\_define} file.

\paragraph{Files.}
The class files are shown below.

{
  \scriptsize
  \begin{tabular}{|l|l|}
    \hline
    file name & description \\
    \hline \hline
    {\tt IMSim/FreezeInteractionHandler.hh}
    & standard C++ header file
    \\ \hline
    {\tt IMSim/FreezeInteractionHandler.cpp}
    & method implementation file
    \\ \hline
  \end{tabular}
}

% -----------------------------------------------------------------------
\subsection{{\tt TrickHLA::Federate}}

\paragraph{Description.}
This class acts as a bridge between the manager and federate ambassador
classes.
It provides basic services for connecting a Trick simulation to an
HLA federation.
It also verifies the simulation data against the FOM and may terminate the
simulation when discrepancies are found.

\paragraph{Files.}
The class files are shown below.
   
{
  \scriptsize
  \begin{tabular}{|l|l|} 
    \hline
    file name & description \\
    \hline \hline
    {\tt TrickHLA/Federate.hh} 
    & standard C++ header file
    \\ \hline
    {\tt TrickHLA/Federate.cpp} 
    & method implementation file
    \\ \hline
  \end{tabular}
}

% -----------------------------------------------------------------------
\subsection{{\tt TrickHLA::FedAmb}}

\paragraph{Description.}
This class implements the HLA {\tt FederateAmbassador} interface.
The methods is implements are callbacks invoked by the HLA runtime
infrastucture.

\paragraph{Files.}
The class files are shown below.
   
{
  \scriptsize
  \begin{tabular}{|l|l|} 
    \hline
    file name & description \\
    \hline \hline
    {\tt TrickHLA/FedAmb.hh} 
    & standard C++ header file
    \\ \hline
    {\tt TrickHLA/FedAmb.cpp} 
    & method implementation file
    \\ \hline
  \end{tabular}
}

% -----------------------------------------------------------------------
\subsection{{\tt TrickHLA::Manager}}

\paragraph{Description.}
This class connects the \TrickHLA\ model to Trick variables.
It handles the sending and receiving of HLA objects, associating them
with the Trick variables specified in the input file.
And it also handles the processing of received interactions.
Much of this logic is delegated to the relevant {\tt TrickHLA::Object}
and {\tt TrickHLA::Interaction} objects.

\paragraph{Files.}
The class files are shown below.
   
{
  \scriptsize
  \begin{tabular}{|l|l|} 
    \hline
    file name & description \\
    \hline \hline
    {\tt TrickHLA/Manager.hh} 
    & standard C++ header file
    \\ \hline
    {\tt TrickHLA/Manager.cpp} 
    & method implementation file
    \\ \hline
  \end{tabular}
}

% -----------------------------------------------------------------------
\subsection{{\tt TrickHLA::ObjectDeleted}}

\paragraph{Description.}
This class is a base class from which simulation developers may
incorporate their own actions to take when an object is deleted from the
federation into the \TrickHLA\ model.
The virtual {\tt delete()} method is a do-nothing implemenation 
in this base class.
The \TrickHLA\ infrastructure invokes the {\tt delete()} method when an object 
is deleted from the federation.


\paragraph{Files.}
The class files are shown below.
   
{
  \scriptsize
  \begin{tabular}{|l|l|} 
    \hline
    file name & description \\
    \hline \hline
    {\tt TrickHLA/ObjectDeleted.hh} 
    & standard C++ header file
    \\ \hline
  \end{tabular}
}

% -----------------------------------------------------------------------
\subsection{{\tt TrickHLA::Conditional}}

\paragraph{Description.}
This class is a base class from which simulation developers may
incorporate the decisions when to send an attribute.
\newline
The virtual {\tt should\_send()} method is a do-nothing implemenation 
in this base class, returning true.
\newline
The \TrickHLA\ infrastructure, on each data cycle, checks for if user wired a
subclass to this attribute. If found, the should\_send() method is called; when
it returns true, this attribute is sent. If not found, the attribute is
always sent.

\paragraph{Files.}
The class files are shown below.

{
  \scriptsize
  \begin{tabular}{|l|l|}
    \hline
    file name & description \\
    \hline \hline
    {\tt TrickHLA/Conditional.hh}
    & standard C++ header file
    \\ \hline
    {\tt TrickHLA/Conditional.cpp}
    & method implementation file
    \\ \hline
  \end{tabular}
}

% -----------------------------------------------------------------------
\subsection{{\tt TrickHLA::Timeline}}

\paragraph{Description.}
This class is a base class from which simulation developers may
provide access to the Scenario Timeline for their simulation.
\newline
The virtual {\tt get\_time()} method is abstract and the user must provide an
implementation when they extend this class.
\newline
The \TrickHLA\ infrastructure will call get\_time() as needed to coordinate
a federation freeze (pause) on the scenario timeline.

\paragraph{Files.}
The class files are shown below.

{
  \scriptsize
  \begin{tabular}{|l|l|}
    \hline
    file name & description \\
    \hline \hline
    {\tt TrickHLA/Timeline.hh}
    & standard C++ header file
    \\ \hline
    {\tt TrickHLA/Timeline.cpp}
    & method implementation file
    \\ \hline
  \end{tabular}
}