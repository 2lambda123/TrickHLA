\chapter{Verification}\label{sec:verification}

This chapter summarizes the verification activities carried out
for each of the \TrickHLA\ requirements.


% ----------------------------------------------------------------
\section{General Requirements}

% -----------------------------
\subsection{TrickHLA\_1: Documentation}
\paragraph{Summary.}
The model must include requirements specifications,
soft\-ware/\-inter\-face/\-version information,
a users guide, and
documentation of test procedures and results.
\paragraph{Method.}Inspection.
\paragraph{Results.}
\TrickHLA\ requirements are documented in
the \href{file:TrickHLAReqt.pdf} {\em \TrickHLA\ Product Requirements}.
The software, interfaces and current version information is documented in
the \href{file:TrickHLASpec.pdf} {\em \TrickHLA\ Product Specification}.
An introduction for users is available in
the \href{file:TrickHLAUser.pdf} {\em \TrickHLA\ User Guide}.
Test procedures and results are documented in this document.


% -----------------------------
\subsection{TrickHLA\_2: Header File Trick Header}
\paragraph{Summary.}
\TrickHLA\ header files must include a Trick header that specifies
the purpose of the file, references, assumptions/limitations and the author.
\paragraph{Method.} Inspection
\paragraph{Results.}
The \TrickHLA\ header files ({\tt .hh} files) are in the
{\tt include/TrickHLA/} directory.
All of them include a Trick file header with the necessary information.

% -----------------------------
\subsection{TrickHLA\_3: Trick Comments for Enumerated Types}
\paragraph{Summary.}
The enumerated values of {\tt enum} types must be accompanied by a comment
explaining each one.
\paragraph{Method.} Inspection.
\paragraph{Results.}
The \TrickHLA\ enumerated types are defined in {\tt include/TrickHLA/Types.hh}.
They are all commented with explanation of what each one means.

% -----------------------------
\subsection{TrickHLA\_4: Trick Comments for Data Structures}
\paragraph{Summary.}
Each data structure must have a Trick-compliant explaining its purpose.
\paragraph{Method.} Inspection.
\paragraph{Results.}
The \TrickHLA\ data structures are C++ classes.
The fields of these classes are defined in the header files
({\tt .hh} files) in the {\tt include/TrickHLA/} directory.
They all have Trick comments.

% -----------------------------
\subsection{TrickHLA\_5: Source File Trick Headers}
\paragraph{Summary.}
\TrickHLA\ source code files ({\tt .c and .cpp} files)
must include a Trick header that specifies
the purpose of the file, references, assumptions/limitations,
Trick job class, library dependencies and the author.
\paragraph{Method.} Inspection.
\paragraph{Results.}
The \TrickHLA\ source code is C++ in {\tt .cpp} files in the
{\tt source/TrickHLA/} directory.
All of the files inlcude a Trick file header with the necessary information.

% -----------------------------
\subsection{TrickHLA\_6: Trick Comments for Function Definitions}
\paragraph{Summary.} Each function must be documented with Trick-compliant
comments that explain the arguments and return value.
\paragraph{Method.} Inspection
\paragraph{Results.}
All the functions defined in the C++ source files ({\tt .cpp} files)
in the {\tt source/TrickHLA/} directory have Trick-compliant comments describing
the arguments and return values.

% -----------------------------
\subsection{TrickHLA\_7: HLA Federate Interface}
\paragraph{Summary.}
\TrickHLA\ must be based on the IEEE 1516.1-2010 service definitions.
\paragraph{Method.} Inspection
\paragraph{Results.}
\TrickHLA\ is built on top of the Pitch HLA system,
which is compliant with IEEE 1516.



% ----------------------------------------------------------------
\section{Data Requirements}

% -----------------------------
\subsection{TrickHLA\_8: Primitive Data Types}
\paragraph{Summary.}
\TrickHLA\ must support a wide variety of C/C++ primitive data types
(e.g, int, long int, float, double, etc...).
\paragraph{Method.} Inspection.
\paragraph{Results.}
HLA saves data in so-called {\em object attributes} and
{\em interaction parameters}.
Inspection of the \TrickHLA\ source files,
{\tt source/TrickHLA/Attribute.cpp}
and
{\tt source/TrickHLA/Parameter.cpp}
reveals support for all the required primitive types.

% -----------------------------
\subsection{TrickHLA\_9: Static Arrays of Primitive Data Types}
\paragraph{Summary.}
\TrickHLA\ must support arrays of the supported primitive types.
\paragraph{Method.} Inspection
\paragraph{Results.}
The {\tt TrickHLA::Attribute} and {\tt TrickHLA::Parameter} classes
both have a method, {\tt get\_\-number\_\-of\_\-items()} which returns
the number of items in an attribute and/or parameters array.
The only type no supported by the code is an HLA logical time.
All the other supported primitive types may occur in attribute or
parameter arrays.

% ----------------------------------------------------------------
\section{Functional Requirements}

% -----------------------------
\subsection{TrickHLA\_10: Data Driven}
\paragraph{Summary.}
\TrickHLA\ must be data driven (i.e., parameterized by values
specified in input files).
\paragraph{Method.} Inspection
\paragraph{Results.}
Like most Trick models, using \TrickHLA\ requires a balance of
jobs defined in {\tt S\_define} files and initialization data
specified in input files.
\TrickHLA\ is aggressively parameterized, allowing many parameters
to be specified in the input files.
The \href{file:TrickHLAUser.pdf} {\em \TrickHLA\ User Guide}
includes a detailed description of these input files
for each of the various \TrickHLA\ capabilities
(e.g., ownership transfer, interactions, publish/subscribe, etc...).

% -----------------------------
\subsection{TrickHLA\_11: HLA Big and Little Endian}
\paragraph{Summary.}
\TrickHLA\ must support big- and little-endian byte ordering.
\paragraph{Method.} Inspection.
\paragraph{Results.}
\TrickHLA\ attribute and parameter primitive type {\em encoding} may be
specified by the developer as
{\tt ENCODING\_BIG\_ENDIAN} or {\tt ENCODING\_LITTLE\_ENDIAN}.
This specification is done by setting a {\tt .rti\_encoding} input
parameter to one of these two values for primitive types.
This can be seen in the source code for the functions,
{\tt TrickHLA::Attri\-bute.\-init\-ial\-ize()}
and
{\tt TrickHLA::Para\-meter.\-init\-ial\-ize()}.

% -----------------------------
\subsection{TrickHLA\_12: HLA Encoding}
\paragraph{Summary.}
\TrickHLA\ must allow strings and/or byte arrays to be encoded as
unicode strings, ASCII strings or opaque data (as defined in the HLA standard).
\paragraph{Method.} Inspection
\paragraph{Results.}
For non-primitive attributes and parameters,
the {\tt .rti\_encoding} input parameter may be specified as
{\tt ENCODING\_C\_STRING}, 
{\tt ENCODING\_UNICODE\_STRING}, 
{\tt ENCODING\_ASCII\_STRING} or
{\tt ENCODING\_OPAQUE\_DATA}.
This can be seen in the source code for the functions,
{\tt TrickHLA::Attribute.initialize()}
and
{\tt TrickHLA::Parameter.initialize()}.


% -----------------------------
\subsection{TrickHLA\_13: Time Advancement}
\paragraph{Summary.}
\TrickHLA\ must support time stamped order HLA services.
\paragraph{Method.} Inspection.
\paragraph{Results.}
Inspection of the {\tt TrickHLA::Federate} class reveals that a federate
built using \TrickHLA\ may be
\begin{itemize}
\item{time regulating (as indicated by the value of the boolean input flag,
  {\tt .time\_regulating}),
}
\item{time constrained (as indicated by the value of the boolean input flag,
  {\tt .time\_constrained}),
}
\item{both, or }
\item{neither.}
\end{itemize}
The HLA time advancement services invoked by \TrickHLA\ are based on the values
of these two flags.

% -----------------------------
\subsection{TrickHLA\_14: Lag Compensation}
\paragraph{Summary.}
\TrickHLA\ must provide optional support for sender- and receiver-side
lag compensation.
\paragraph{Method.} Inspection
\paragraph{Results.}
The class, {\tt TrickHLA::LagCompensation}, provides this capability.
It is not required, but may be used for sender- or receiver-side
compensation.
The \href{file:TrickHLAUser.pdf} {\em \TrickHLA\ User Guide}
discusses this class in more detail.

% -----------------------------
\subsection{TrickHLA\_15: Interactions}
\paragraph{Summary.}
\TrickHLA\ must support sending and receiving of interactions in
receive order (RO) or time stamp order (TSO).
\paragraph{Method.} Inspection
\paragraph{Results.}
The class, {\tt TrickHLA::InteractionHandler}, provides this capability.
It defines two {\tt send\_interaction()} methods, one of which is used to send receive
order interactions and the other of which is used to send time stamp order
interactions with some specified timetag.
The class also defines a virtual method (which may be overridden in
subclasses) which is invoked automatically whenever interactions
(RO or TSO) arrive.
The \href{file:TrickHLAUser.pdf} {\em \TrickHLA\ User Guide}
discusses this class in more detail.

% -----------------------------
\subsection{TrickHLA\_16: Ownership Transfer}
\paragraph{Summary.}
\TrickHLA\ must provide support for HLA ownership transfer.
\paragraph{Method.} Inspection
\paragraph{Results.}
The class, {\tt TrickHLA::OwnershipHandler}, provides this capability.
it provides several {\tt push\_ownership()} methods
that result in the federate {\em divesting} itself of ownership
for the relevant attribute (only if the federate is the attribte's owner).
The class also provides several {\tt pull\_ownership()} methods
that result in the federate {\em acquiring} ownership of the relevant
attribute if it has been divested by its owner.
The \href{file:TrickHLAUser.pdf} {\em \TrickHLA\ User Guide}
discusses this class in more detail.

% -----------------------------
\subsection{TrickHLA\_17: Dynamic Initialization}
\paragraph{Summary.}
\TrickHLA\ must support dynamic initialization of an HLA federation
in which the federates may exchange data before the simulation begins.
\paragraph{Method.} Inspection
\paragraph{Results.}
NOTE: This section needs to be rewriten to not reference the DSES code.

\TrickHLA\ supports this via the {\em multiphase initialization process}.
This process is defined in
{
  \href{file:DSES\_Multiphase\_Init\_Design\_Document.pdf}
  {\em
    Distributed Space Exploration Simulation Multiphase Initialization Design
  }
}
\cite{trickhlaenv:DSES-multiphase-init-design}.
The machinery supporting this capability is exposed in the
{\tt TrickHLA::ExecutionControlBase} class, which has an input parameter consisting
of a comma-separated list of synchronization point names, each of which
corresponds to a different phase in the initialization process.
The \href{file:TrickHLAUser.pdf} {\em \TrickHLA\ User Guide}
discusses how to construct a Trick {\tt S\_define} file
that schedules initialization jobs for execution during each phase of this
processes.

% -----------------------------
\subsection{TrickHLA\_18: Automatic Simulation Startup}
\paragraph{Summary.}
\TrickHLA\ must provide a mechanism for the various federates to
synchronize with each other (i.e., for all of them to arrive)
before the simulation begins in earnest.
\paragraph{Method.} Inspection
\paragraph{Results.}
NOTE: This section needs to be rewriten to reference the
TrickHLA::ExecutionControlBase class and associated implementations.

The {\tt TrickHLA::Federate} class has several input parameters that
may be used to specify a list of federates which must join the federation
before the execution begins:
\begin{itemize}
\item{
  {\tt .enable\_known\_feds}, which enables/disables this feature,
}
\item{
  {\tt .known\_feds\_count}, which specifies how many federations are
  to be governed by this mechanism,
}
\item{
  {\tt .known\_feds}, which is an array of size
  {\tt .known\_feds\_count} of structures, each of which specifies
  the name of the federation and whether or not is must be present
  before the federation execution may begin.
}
\end{itemize}
The \href{file:TrickHLAUser.pdf} {\em \TrickHLA\ User Guide}
presents several examples in which this capability
is used.

In addition to this capability, the {\tt TrickHLA::ExecutionCoontorlBase}
class provides a similar capability.
The class has a parameter, {\tt .required\_federates}, which is a
comma-separated list of the names of the federates that must be
present in order for the federation execution to begin.

% -----------------------------
\subsection{TrickHLA\_19: Pack / Unpack of Simulation Data}
\paragraph{Summary.}
\TrickHLA\ must provide a mechanism for user specified code to be called to
perform processing of data sent to or received from the HLA interface.
\paragraph{Method.} Inspection
\paragraph{Results.}
The class, {\tt TrickHLA::Packing}, provides the capability. It provides a
pack() method that is called before data is sent through the HLA interface.  
It also provides an unpack() method that is called when data is received 
through the HLA interface.
The \href{file:TrickHLAUser.pdf} {\em \TrickHLA\ User Guide}
discusses this class in detail.

% -----------------------------
\subsection{TrickHLA\_20: ObjectDeleted Callback}
\paragraph{Summary.}
\TrickHLA\ must provide a mechanism for notification of an object being 
deleted from the federation.
\paragraph{Method.} Inspection
\paragraph{Results.}
The class, {\tt TrickHLA::ObjectDeleted}, provides the capability. It 
provides a deleted() method that is called when an object is deleted from 
the federation.
The \href{file:TrickHLAUser.pdf} {\em \TrickHLA\ User Guide}
discusses this class in detail.

% -----------------------------
\subsection{TrickHLA\_21: Federation Restore Callback}
\paragraph{Summary.}
\TrickHLA\ must provide a mechanism for a trick model to request a federation
restore from the {\tt RTI}.
\paragraph{Method.} Inspection
\paragraph{Results.}
The class, {\tt TrickHLA::Federate}, provides the capability. It
provides a perform\_restore() method that sends a completed federation
restore request to the {\tt TrickHLA::Manager} which, in turn, sends the
request to the {\tt RTI}.
The \href{file:TrickHLAUser.pdf} {\em \TrickHLA\ User Guide}
discusses this class in detail.

% -----------------------------
\subsection{TrickHLA\_22: Federation Save Callback}
\paragraph{Summary.}
\TrickHLA\ must provide a mechanism for a trick model to request a federation
save from the {\tt RTI}.
\paragraph{Method.} Inspection
\paragraph{Results.}
The class, {\tt TrickHLA::Manager}, provides the capability. It
provides a start\_federation\_save(), start\_federation\_save\_at\_sim\_time() and
start\_federation\_save\_at\_scenario\_time() methods that starts the federation
wide save process.
The \href{file:TrickHLAUser.pdf} {\em \TrickHLA\ User Guide}
discusses this class in detail.

% -----------------------------
\subsection{TrickHLA\_23: Conditional sending of attributes}
\paragraph{Summary.}
\TrickHLA\ must provide a mechanism for a trick model to conditionally send
attributes over the wire.
\paragraph{Method.} Inspection
\paragraph{Results.}
The class, {\tt TrickHLA::Conditional}, provides the capability. It provides a
should\_send() method that is called on each send cycle to identify if an
attribute should be sent over the wire.
The \href{file:TrickHLAUser.pdf} {\em \TrickHLA\ User Guide}
discusses this class in detail.

% -----------------------------
\subsection{TrickHLA\_24: Multiple verbose levels}
\paragraph{Summary.}
\TrickHLA\ must provide a mechanism to print multiple levels of information from
the \TrickHLA\ software. It also must provide a mechanism to allow the user to
specify which \TrickHLA\ module(s) shall print messages.
\paragraph{Method.} Inspection
\paragraph{Results.}
The class, {\tt TrickHLA::DebugHandler}, provides this capability. It provides a
print() method, accepting a debug level and code section, returns true
or false after determining if the message should be printed.
The \href{file:TrickHLASpec.pdf} {\em \TrickHLA\ Product Specification}
discusses this class in detail.
