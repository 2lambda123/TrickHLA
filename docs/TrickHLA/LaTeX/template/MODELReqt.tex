%%%%%%%%%%%%%%%%%%%%%%%%%%%%%%%%%%%%%%%%%%%%%%%%%%%%%%%%%%%%%%%%%%%%%%%%%
%
% File: MODELReqt.tex
%
% Purpose: MODEL Product Requirements
%
%%%%%%%%%%%%%%%%%%%%%%%%%%%%%%%%%%%%%%%%%%%%%%%%%%%%%%%%%%%%%%%%%%%%%%%%%

\newcommand\documentHistory{
{\bf Author} & {\bf Date} & {\bf Description} \\ \hline \hline
YOUR NAME & DATE & Initial Version \\ \hline
}

\newcommand\DocumentChangeHistory{
{\bf Revised by} & {\bf Date} & {\bf Description} \\ \hline \hline

% This documentation file's change history includes:
% "REVISED BY name(s)" & "revision DATE" & "revision DESCRIPTION"
%  ------------------     -------------     --------------------
                       &                 &                    \\ \hline
}

\documentclass[twoside,11pt,titlepage]{report}

%
% Bring in the AMS math environment
%
\usepackage{amsmath}

%
% Bring in the common page setup
%
\usepackage{trickhlaenv}

%
% Bring in the common math nomenclature
%
\usepackage{trickhlamath}

%
% Bring in the model-specific commands
%
\usepackage{MODEL}

%
% Bring in the graphics environment
%
\usepackage{graphicx}

%
% Bring in the hyper ref environment
%
\usepackage[colorlinks]{hyperref}
%  keywords for pdfkeywords are separated by commas
\hypersetup{
   pdftitle={\MODEL\ Product Requirements},
   pdfauthor={YOUR NAME},
   pdfkeywords={\MODEL, Product Requirements},
   pdfsubject={\MODEL\ Product Requirements}}

\begin{document}

%%%%%%%%%%%%%%%%%%%%%%%%%%%%%%%%%%%
% Front matter
%%%%%%%%%%%%%%%%%%%%%%%%%%%%%%%%%%%
\pagenumbering{roman}

\docid{DD.mm.02}
\docrev{1.0}
\date{DATE}
\modelname{\MODEL}
\doctype{Product Requirements}
\author{YOUR NAME}
\managers{
  Dan E. Dexter \\ Project Manager \\
  Leslie J. Quiocho \\ Common Model Lead \\
  Rodney M. Ondler \\ Software, Robotics, and Simulation Division Chief}
\pdfbookmark{Title Page}{titlepage}
\makeTrickhlaenvTitlepage

\pdfbookmark{Abstract}{abstract}
%%%%%%%%%%%%%%%%%%%%%%%%%%%%%%%%%%%%%%%%%%%%%%%%%%%%%%%%%%%%%%%%%%%%%%%%%
%
% Purpose: Abstract for MODEL
%
% Author: YOUR NAME - DATE
% 
% Modified: 
%  
%
%%%%%%%%%%%%%%%%%%%%%%%%%%%%%%%%%%%%%%%%%%%%%%%%%%%%%%%%%%%%%%%%%%%%%%%%%

\begin{abstract}
This is the abstract of the \MODEL.
\end{abstract}


\pdfbookmark{Contents}{contents}
\tableofcontents
\vfill

\pagebreak

%%%%%%%%%%%%%%%%%%%%%%%%%%%%%%%%%%%
% Main Document Body
%%%%%%%%%%%%%%%%%%%%%%%%%%%%%%%%%%%
\pagenumbering{arabic}

%----------------------------------
\chapter{Introduction}\label{sec:intro}
%----------------------------------

%%%%%%%%%%%%%%%%%%%%%%%%%%%%%%%%%%%%%%%%%%%%%%%%%%%%%%%%%%%%%%%%%%%%%%%%%
%
% Purpose: Introduction for MODEL
%
% Author: YOUR NAME - DATE 
% 
% Modified: 
%  
%
%%%%%%%%%%%%%%%%%%%%%%%%%%%%%%%%%%%%%%%%%%%%%%%%%%%%%%%%%%%%%%%%%%%%%%%%%

\MODEL\ introduction. 


\section{Identification of Document}
This document describes the as-written requirements on the \MODEL\
developed for use in the Trick Simulation Environment.
This document adheres to the documentation standards defined in
NASA Software Engineering Requirements Standard \cite{NASA:SWE}.

\section{Scope of Document}
This document provides information on the requirements for
the \MODEL.

\section{Purpose and Objectives of Document}
The TrickHLA package, which includes the \MODEL,
is classified as a Class C (Mission Support Software)
product per NASA NPR 7150.2\cite{NASA:SWE}
to enable the use of the package in a critical
application such as a dynamic simulation
used to certify the flight readiness of a space vehicle.

The purpose of this document is to define the set
of requirements that the \MODEL\ must achieve to enable
its use in Class C products.

\section{Documentation Status and Schedule}
The information in this document is current with the \TrickHLAid\
implementation of the \MODEL. Updates will be kept current with
module changes.

\begin{tabular}{||l|l|l|} \hline
\documentHistory
\end{tabular}

\begin{tabular}{||l|l|l|} \hline
\DocumentChangeHistory
\end{tabular}

\section{Document Contents}
This document is organized into the following sections:

\begin{description}

\item[Chapter \ref{sec:intro}: Introduction] -
Identifies this document, defines the scope and purpose, present status,
and provides a description of each major section.

\item[Chapter \ref{sec:docs}: Related Documentation] -
Lists the related documentation that is applicable to this project.

\item[Chapter \ref{sec:reqts}: Requirements] -
Presents the requirements for the \MODEL.

\item[Bibliography] -
Informational references associated with this document.

\end{description}

\chapter{Related Documentation}\label{sec:docs}

\section{Parent Documents}
The following documents are parent to this document:

\begin{itemize}
\item{\href{file:\TRICKHLAHOME/docs/TrickHLA.pdf}
           {\em Trick High Level Architecture (\TrickHLA)}}
\cite{trickhlaenv:TrickHLA}

\item{\href{file:MODEL.pdf}
           {\em \MODEL}}
\cite{trickhlaenv:MODEL}
\end{itemize}

\section{Applicable Documents}
The following documents are referenced herein and are directly
applicable to this document:

\begin{itemize}
\item{\href{file:MODELSpec.pdf}
           {\em \MODEL\ Product Specification}}
\cite{trickhlaenv:MODELSpec}

\item{\href{file:MODELUser.pdf}
           {\em \MODEL\ User Guide}}
\cite{trickhlaenv:MODELUser}

\item{\href{file:MODELIVV.pdf}
           {\em \MODEL\ Inspection, Verification, and Validation}}
\cite{trickhlaenv:MODELIVV}

\item{\em The Trick User's Guide: Trick 2005.0 Release}
\cite{Vetter:TrickUser}

\item{\em Trick Simulation Environment: User Training Materials:
          Trick 2005.0 Release}
\cite{Vetter:TrickUTM}

\item{\em Trick Simulation Environment: Version Description:
          Trick 2005.0 Release}
\cite{Vetter:TrickVD}

\item{\em The Trick Design Document: Trick 2005.0 Release}
\cite{Vetter:TrickDD}

\item{\em NASA Software Engineering Requirements}
\cite{NASA:SWE}

\end{itemize}


\chapter{Requirements}\label{sec:reqts}

\section{General Requirements}\label{sec:general_reqts}

This section identifies general requirements for the \MODEL.


\requirement{Documentation}
\label{reqt:documentation}
\begin{description}
  \item[Requirement:]\ \newline
    The documentation for the model shall include

    \subrequirement{}
    \label{reqt:reqts_doc}
      Software requirements specification.

    \subrequirement{}
    \label{reqt:design_doc}
      Software, interface, and software version descriptions.

    \subrequirement{}
    \label{reqt:test_doc}
      Software test procedures and results.

    \subrequirement{}
    \label{reqt:user_doc}
      User Guide.

  \item[Rationale:]\ \newline
    The listed items are needed to comply with NASA NPR 7150.2
    as a Class C product.

  \item[Verification:]\ \newline
    Inspection
\end{description}


\requirement{Header File Trick Header}
\label{reqt:h_trick_header}
\begin{description}
  \item[Requirement:]\ \newline
    All header files associated with the model shall have an appropriate
    Trick header. The Trick header for a header file shall include

    \subrequirement{Purpose}\label{reqt:h_trick_header_purpose}
      A brief description of the file.

    \subrequirement{References}\label{reqt:h_trick_header_refs}
      A list of applicable references that describe the model.

    \subrequirement{Assumptions and limitations}
    \label{reqt:h_trick_header_assum}
      A list of the assumptions made in developing the model and
      any limitations on the use of the model.

    \subrequirement{Programmer}\label{reqt:h_trick_header_prog}
      A list of the developers who created or modified the file.

  \item[Rationale:]\ \\[-20pt]
    \begin{itemize}
      \item The Trick header in a header file
        indicates that Trick should process the file.
      \item Properly documenting the TrickHLA package models
        is a key goal of the TrickHLA verification,
        validation, and documentation task.
      \item Maintaining a version history is good programming
        technique and is mandatory per NPR 7150.2.
    \end{itemize}

  \item[Verification:]\ \newline
    Inspection
\end{description}

\requirement{Trick Comments for Enumerated Types}
\label{reqt:enum_trick_comments}
\begin{description}
  \item[Requirement:]\ \newline
    Each tag defined in a enumeration type in a model header
    file {\em should} have a comment describing the tag that follows
    the tag declaration.

  \item[Rationale:]\ \newline
    Short tag names may not suffice in establishing
    the meaning of the tag.

  \item[Verification:]\ \newline
    Inspection \newline
    Enumerated types that fail to meet this optional requirement
    shall be noted as such in the model verification document.
\end{description}


\requirement{Trick Comments for Data Structures}
\label{reqt:struct_trick_comments}
\begin{description}
  \item[Requirement:]\ \newline
    Each element of a data structure defined in a model header
    file shall have a Trick-compliant comment describing the
    element that follows the element declaration.

  \item[Rationale:]\ \newline
    The element comment is required by Trick.

  \item[Verification:]\ \newline
    Inspection
\end{description}


\requirement{Source File Trick Headers}
\label{reqt:c_trick_header}
\begin{description}
  \item[Requirement:]\ \newline
    Each externally visible function defined in the source files
    associated with the model shall have an appropriate Trick header.
    The Trick header for a function shall include

    \subrequirement{Purpose}\label{reqt:c_trick_header_purpose}
      A brief description of the function.

    \subrequirement{References}\label{reqt:c_trick_header_refs}
      A list of applicable references that describe the function.

    \subrequirement{Assumptions and limitations}
    \label{reqt:c_trick_header_assum}
      A list of the assumptions made in developing the function and
      any limitations on the use of the function.

    \subrequirement{Class}
    \label{reqt:c_trick_header_class}
      The default Trick job classification of the function.

    \subrequirement{Library dependency}
    \label{reqt:c_trick_header_depend}
      A list of the object files upon which the function depends,
      starting with the current file.

    \subrequirement{Programmer}\label{reqt:c_trick_header_prog}
      A list of the developers who created or modified the file.

  \item[Rationale:]\ \\[-20pt]
    \begin{itemize}
      \item The Trick header that precedes a function
        indicates that the function is available for
        use in a simulation {\em S\_define} file.
      \item Properly documenting the TrickHLA package models
        is a key goal of the TrickHLA verification,
        validation, and documentation task.
      \item Maintaining a version history is good programming
        technique and is mandatory per NPR 7150.2.
    \end{itemize}

  \item[Verification:]\ \newline
    Inspection
\end{description}


\requirement{Trick Comments for Function Definitions}
\label{reqt:func_trick_comments}
\begin{description}
  \item[Requirement:]\ \newline
    Each function shall be commented with a Trick-compliant
    set of comments that describe the return value from the
    function and that describe the nature of the arguments
    passed to the function.

  \item[Rationale:]\ \newline
    The function definition comments are required by Trick.

  \item[Verification:]\ \newline
    Inspection
\end{description}


ADD ANY OTHER GENERAL REQUIREMENTS THAT APPLY TO THE
MODEL AS A WHOLE.


\section{Data Requirements}\label{sec:data_reqts}

This section identifies requirements on the data
represented by the \MODEL. These as-built requirements are
based on the \MODEL\ data definition header files.

EXAMPLE, universal time, not complete:

\requirement{Time Representation}
\label{reqt:data_time_representation}
\begin{description}
  \item[Requirement:]\ \newline
    The universal time model shall represent time in the
    each of the following time systems:

    \subrequirement{TAI}\label{reqt:data_time_rep_TAI}
      \ \newline
      International Atomic Time,
      a very accurate and stable time scale calculated as a weighted
      average of the time kept by about 200 cesium atomic clocks in
      over 50 national laboratories worldwide.

    \subrequirement{UT1}\label{reqt:data_time_rep_UT1}
      \ \newline
      Universal Time, a measure of the rotation angle of the Earth
      as observed astronomically.

    \subrequirement{UTC}\label{reqt:data_time_rep_UTC}
      \ \newline
      Coordinated Universal Time, the basis for the worldwide system
      of civil time.

    \subrequirement{and others}
      \ \newline
      THIS IS AN INCOMPLETE EXAMPLE.

  \item[Rationale:]\ \newline
    The purpose of the universal time module is to represent
    time in the multiplicity of time scales that are expected to be
    need by various simulation developers.

  \item[Verification:]\ \newline
    Inspection, Test
\end{description}


\section{Functional Requirements}\label{sec:func_reqts}

This section identifies requirements on the functional
capabilities provided by the \MODEL. These as-built requirements are
based on the \MODEL\ source files.

EXAMPLE, universal time, not complete:

\requirement{Time Initialization}
\label{reqt:func_time_initialization}
\begin{description}
  \item[Requirement:]\ \newline
    The universal time model shall be capable of initializing
    time using any one of the following mechanisms:

    \subrequirement{incomplete}
      THIS IS AN INCOMPLETE EXAMPLE.

  \item[Rationale:]\ \newline
    The purpose of the universal time module is to represent
    time in the multiplicity of time scales that are expected to be
    need by various simulation developers.

  \item[Verification:]\ \newline
    Test
\end{description}

\requirement{Time Update}
\label{reqt:func_time_update}
\begin{description}
  \item[Requirement:]\ \newline
    \subrequirement{}
    The universal time model shall correctly update all represented
    time values as the simulation clock advances.

    \subrequirement{}
    The passage of one second of simulation time shall correspond
    to the passage of one second of International Atomic Time.

  \item[Rationale:]\ \newline
    The purpose of the universal time module is to represent
    time in the multiplicity of time scales that are expected to be
    need by various simulation developers.

  \item[Verification:]\ \newline
    Test
\end{description}

%%%%%%%%%%%%%%%%%%%%%%%%%%%%%%%%%%%%%%%%%%%%%%%%%%%%%%%%%%%%%%%%%%%%%%%%%
% Bibliography
%%%%%%%%%%%%%%%%%%%%%%%%%%%%%%%%%%%%%%%%%%%%%%%%%%%%%%%%%%%%%%%%%%%%%%%%%
\newpage
\pdfbookmark{Bibliography}{bibliography}
\bibliography{trickhlaenv,MODEL}
\bibliographystyle{plain}

\end{document}
