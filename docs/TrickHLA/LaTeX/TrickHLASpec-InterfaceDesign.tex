%%%%%%%%%%%%%%%%%%%%%%%%%%%%%%%%%%%%%%%%%%%%%%%%%%%%%%%%%%%%%%%%%%%%%%%%%
%
% File: TrickHLASpec-InterfaceDesign.tex
%
% Purpose: TrickHLA Product Specification - Interface Design chapter
%
%%%%%%%%%%%%%%%%%%%%%%%%%%%%%%%%%%%%%%%%%%%%%%%%%%%%%%%%%%%%%%%%%%%%%%%%%

\chapter{Interface Design}\label{sec:interface_design}

This chapter introduces the main interfaces to the \TrickHLA\ model.
Since the model is intended to be used in a Trick simulation,
these interfaces are all Trick-related.

%%%%%%%%%%%%%%%%%%%%%%%%%%%%%%%%%%%%%%%%%%%%%%%%%%%%%%%%%%%%%%%%%%%%%%%%%
\section{Summary}

The main interface between the \TrickHLA\ model and Trick is the
default {\ttfamily sim\_object}.
Inclusion of this {\ttfamily sim\_object} in an {\ttfamily S\_define} file
results in the simulation automatically joining a distributed HLA federation,
sending certain specified simulation data to the federation as they change,
and receiving certain changing data from other simulations in the federation.
The details of this process
(i.e., which federation to join, which data to send, and which data to
subscribe to) are configured through standard Trick input files.

Interfaces to other HLA capabilities
(e.g., interactions and ownership management)
are through \TrickHLA\ classes and their methods, which may be invoked as
Trick jobs or directly from simulation-specific code.

This section summarizes these interfaces.

%%%%%%%%%%%%%%%%%%%%%%%%%%%%%%%%%%%%%%%%%%%%%%%%%%%%%%%%%%%%%%%%%%%%%%%%%
\section{The Main \TrickHLA\ Interface}

The main \TrickHLA\ interface is the default {\ttfamily sim\_object} that
is distributed with the model.
This {\ttfamily sim\_object} contains data and jobs that handle
\begin{itemize}
  \item{initialization of the HLA infrastructure},
  \item{joining the specified federation and waiting for all expected federates to join},
  \item{sending simulation data to other federates},
  \item{receiving simulation data from other federates},
  \item{handling the HLA time advancement logic} and
  \item{saving the federation}.
\end{itemize}

Using this default {\ttfamily sim\_object} involves two steps:
(a) inserting the object in the simulation's {\ttfamily S\_define} file, and
(b) specifying the relevant configuration data in Trick input files.
These steps are discussed below.

% -----------------------------------------------------------------------
\subsection{The \TrickHLA\ {\ttfamily sim\_object}}

In most cases a Trick simulation may be integrated with HLA by simply
inserting the default \TrickHLA\ {\ttfamily sim\_object} to the simulation's
{\ttfamily S\_define} file and then configuring the object appropriately
using Trick input files.
An abbreviated version of the {\ttfamily sim\_object} is shown in
Figure~\ref{fig:default-sim-object}.

\begin{figure}[th]
  \begin{center}
    \scriptsize
    \begin{verbatim}
sim_object {
   TrickHLA: TrickHLAFreezeInteractionHandler freeze_ih;
   TrickHLA: TrickHLAFedAmb   federate_amb;
   TrickHLA: TrickHLAFederate federate;
   TrickHLA: TrickHLAManager  manager;
   double checkpoint_time;
   char   checkpoint_label[256];

   P1 (initialization) TrickHLA: THLA.manager.print_version();

   P1 (initialization) TrickHLA: THLA.federate.fix_FPU_control_word();

   P60 (initialization) TrickHLA: THLA.federate_amb.initialize(
      In TrickHLAFederate * federate = &THLA.federate,
      In TrickHLAManager  * manager  = &THLA.manager );

   P60 (initialization) TrickHLA: THLA.federate.initialize(
      Inout TrickHLAFedAmb * federate_amb = &THLA.federate_amb );
   P60 (initialization) TrickHLA: THLA.manager.initialize(
      In TrickHLAFederate * federate = &THLA.federate );

   P65534 (initialization) TrickHLA: THLA.manager.initialization_complete();

   P65534 (initialization) TrickHLA: THLA.federate.check_pause_at_init( 
      In const double check_pause_delta = THLA_CHECK_PAUSE_DELTA );

   P1 (checkpoint)          TrickHLA: THLA.federate.setup_checkpoint();
   (freeze)                 TrickHLA: THLA.federate.perform_checkpoint();
   P1 (pre_load_checkpoint) TrickHLA: THLA.federate.setup_restore();
   (freeze)                 TrickHLA: THLA.federate.perform_restore();

   (freeze)   TrickHLA: THLA.federate.check_freeze();
   (unfreeze) TrickHLA: THLA.federate.exit_freeze();

   P1 (THLA_DATA_CYCLE_TIME, environment) TrickHLA: THLA.federate.wait_for_time_advance_grant();

   P1 (THLA_INTERACTION_CYCLE_TIME, environment) TrickHLA: THLA.manager.process_interactions();

   P1 (THLA_DATA_CYCLE_TIME, environment) TrickHLA: THLA.manager.process_deleted_objects();

   P1 (0.0, environment) TrickHLA: THLA.manager.start_federation_save(
      In const char * file_name = THLA.checkpoint_label );

   P1 (0.0, environment) TrickHLA: THLA.manager.start_federation_save_at_sim_time(
      In double freeze_sim_time = THLA.checkpoint_time,
      In const char * file_name = THLA.checkpoint_label );

   P1 (0.0, environment) TrickHLA: THLA.manager.start_federation_save_at_scenario_time(
      In double freeze_sim_time = THLA.checkpoint_time,
      In const char * file_name = THLA.checkpoint_label );

   P1 (THLA_DATA_CYCLE_TIME, environment) TrickHLA: THLA.manager.receive_cyclic_data();

   P65534 (THLA_DATA_CYCLE_TIME, logging) TrickHLA: THLA.manager.send_cyclic_and_requested_data();

   P65534 (THLA_DATA_CYCLE_TIME, logging) TrickHLA: THLA.manager.process_ownership();

   P65534 (THLA_DATA_CYCLE_TIME, logging) TrickHLA: THLA.federate.time_advance_request();

   P65534 (THLA_DATA_CYCLE_TIME, logging) TrickHLA: THLA.federate.check_freeze_time();

   P65534 (THLA_DATA_CYCLE_TIME, THLA_CHECK_PAUSE_JOB_OFFSET, logging) TrickHLA: THLA.federate.check_pause( 
      In const double check_pause_delta = THLA_CHECK_PAUSE_DELTA );

   P65534 (THLA_DATA_CYCLE_TIME, THLA_CHECK_PAUSE_JOB_OFFSET, logging) TrickHLA: THLA.federate.enter_freeze();

   P65534 (shutdown) TrickHLA: THLA.manager.shutdown();
} THLA;
    \end{verbatim}
  \end{center}
\caption{The Default \TrickHLA\ {\ttfamily sim\_object}}
\label{fig:default-sim-object}
\end{figure}

The contents of this {\ttfamily sim\_object} include:
\begin{itemize}
  \item{data declarations,}
  \item{scheduled jobs,}
  \item{initialization jobs,}
  \item{freeze related jobs, and}
  \item{shutdown jobs.}
\end{itemize}

The data declarations are partially explained in the discussion of
input files, below.\footnote{
Other than knowing how to initialize the significant components of these data,
there is no need to understand their internal structure in detail.
}
The initialization jobs handle the initialization of the internal \TrickHLA\
classes and need not concern most simulation developers.
The freeze related jobs are included so that freezing a Trick job does not
completely disable underlying HLA-related threads.
And the shutdown jobs handle resignation from the HLA federation when the
Trick simulation completes.

Other than the requirement that certain data be initialized consistently with
the needs of the simulation, this {\ttfamily sim\_object} may be treated as
a black box -- i.e., there is usually no need to modify it after copying it
verbatim into the simulation's {\ttfamily S\_define} file.
The following section elaborates on how to appropriately set the input data.

% -----------------------------------------------------------------------
\subsection{Input Data Files}

The various data structures declared in the \TrickHLA\ default
{\ttfamily sim\_object} must be initialized through the Trick
simulation input file.
The following sections discuss the various initialization parameters.

% --------------------------------------
\subsubsection{HLA Runtime and Federation-Related Parameters}

This section describes the input parameters necessary to connect the
simulation to the HLA runtime infrastructure and other data used to set up
the federate.
The parameters are discussed in Table~\ref{tab:runtime-and-federation-parameters}.

\begin{table}[t]
  \scriptsize
  \begin{center}
    \begin{tabular}{|l|l|p{3.25in}|}
    \hline
    parameter name & type & description \\
    \hline \hline
      {\ttfamily THLA.federate.local\_settings} & string
      & RTI vendor specific string that specifies the CRC hostname or IP address with port number where the HLA runtime executive is running
      \\
      \hline
      {\ttfamily THLA.federate.name} & string
      & federate name to be assigned to the simulation
      \\
      \hline
      {\ttfamily THLA.federate.enable\_FOM\_validation} & true or false
      & true to enable FOM validation or false to disable it (default: true)
      \\
      \hline
      {\ttfamily THLA.federate.FOM\_modules} & string
      & name of the FOM file when using the IEEE 1516-2000 and SISO-STD-004-2004 standards, or a comma separated list of FOM-module filenames when using the HLA Evolved IEEE 1516-2010 standard
      \\
      \hline
      {\ttfamily THLA.federate.MIM\_module} & string
      & name of the MOM and Initialization Module (MIM) file for the HLA Evolved IEEE 1516-2010 standard
      \\
      \hline
      {\ttfamily THLA.federate.federation\_name} & string
      & name of the federation to join
      \\
      \hline
      {\ttfamily THLA.federate.enable\_known\_feds} & true or false
      & should the simulation wait for all the other (known) federates
        in the federation before it begins?
      \\
      \hline
      {\ttfamily THLA.federate.known\_feds\_count} & integer
      & how many known federations are there?
      \\
      \hline
      {\ttfamily THLA.federate.known\_feds} & alloc($N_{feds}$)
      & allocate an array with $N_{feds}$ elements, where $N_{feds}$ is the value of
        {\ttfamily THLA.federate.known\_feds\_count}
      \\
      \hline
      {\ttfamily THLA.federate.known\_feds[$i$].name} & string
      & the name of known federate $i \in 0...N_{feds}-1$ where $N_{feds}$ is the value of
        {\ttfamily THLA.federate.known\_feds\_count}
      \\
      \hline
      {\ttfamily THLA.federate.known\_feds[$i$].\-required} & true or false
      & is federate $i \in 0...N_{feds}-1$ required to be present before
        this federate begins, where $N_{feds}$ is the value of
        {\ttfamily THLA.\-trick\_fed.\-known\_feds\_count}
      \\
      \hline
      {\ttfamily THLA.federate.use\_preset\_master} & true or false
      & true to force the use of the preset THLA.federate.master federate flag
        value or false to automatically determine the master federate during
        the multiphase initialization process (default: false)
      \\
      \hline
      {\ttfamily THLA.federate.master} & true or false
      & true when this federate is the master federate for the multiphase
        initialization process. When THLA.federate.use\_preset\_master is set to
        true then the THLA.federate.master flag is used to preset the master
        federate, otherwise the THLA.federate.master flag is ignored
        (default: false)
      \\
      \hline
      {\ttfamily THLA.federate.can\_rejoin\_federation} & true or false
      & true to signal this federate to resign from the federation in a manner
        which makes rejoining the running federation possible. See Section 16
        of the \href{file:\TRICKHLAHOME/docs/IMSim_Multiphase_Init_Design_Document.pdf}
        {IMSim\_Multiphase\_Init\_Design\_Document.pdf} for more details.
        (default: false)
      \\
      \hline
      {\ttfamily THLA.federate.scenario\_timeline} & TrickHLA::Timeline
      & A pointer to a TrickHLA::Timeline implementation for a scenario timeline.
        If nothing is specified \TrickHLA\ will use the TrickHLA::SimTimeline
        implementation, which uses the Trick simulation time as the default
        scenario timeline and is only valid if all federates are Trick based
        simulations.
        (default: TrickHLA::SimTimeline)
      \\
      \hline
      {\ttfamily THLA.federate.multiphase\_init\_sync\_points} & string
      & A comma-separated list of the simulation specific multiphase initialization
        synchronization-point labels as specified in the Federation Agreement.
      \\
      \hline
    \end{tabular}
  \end{center}
  \caption{HLA Runtime and Federation-Related Parameters}
  \label{tab:runtime-and-federation-parameters}
\end{table}

% --------------------------------------
\subsubsection{Time-Related Parameters}

This section describes the time-related input parameters.
This includes parameter that govern HLA time management
as well as parameters related to whether or not
the simulation attempts to keep HLA simulation time synchronized with
the ``wallclock.''
The parameters are discussed in Table~\ref{tab:time-parameters}.

\begin{table}[h]
  \scriptsize
  \begin{center}
    \begin{tabular}{|l|l|p{3.25in}|}
    \hline
    parameter name & type & description \\
    \hline \hline
      {\ttfamily THLA.federate.lookahead\_time} & float
      & lookahead time for HLA time management
      \\
      \hline
      {\ttfamily THLA.federate.time\_regulating} & true or false
      & is this a {\em time regulating} federate?
      \\
      \hline
      {\ttfamily THLA.federate.time\_constrained} & true or false
      & is this a {\em time constrained} federate?
      \\
      \hline
      {\ttfamily THLA.federate.time\_management} & true or false
      & enables / disables HLA time management
      \\
      \hline
      {\ttfamily sys.exec.in.rt\_software\_frame} & float
      & realtime frame (in seconds).
        The Trick timing loop will synchronize HLA simulation time
        with the wallclock at the end of this interval.
        Usually that means suspending the simulation until the
        specified interval has elapsed on the wallclock.
        Set this to a very large number to disable realtime execution.
      \\
      \hline
      {\ttfamily sys.exec.in.rt\_itimer\_frame} & float
      & itimer frame (in seconds).
        If realtime is enabled, set this value to the same value as
        sys.exec.in.rt\_software\_frame.
      \\
      \hline
      {\ttfamily sys.exec.in.rt\_timer} & Yes or No
      & should Trick use itimers when synchronizing HLA simulation time
        with the ``wallclock''?
        To avoid a spin lock (which consumes CPU cycles) while the
        simulation waits for simulation/wallclock time synchronization,
        this value should be set to Yes.
      \\
      \hline
    \end{tabular}
  \end{center}
  \caption{Time-Related Parameters}
  \label{tab:time-parameters}
\end{table}

% --------------------------------------
\subsubsection{Object and Attribute-Related Parameters}

This section describes the input parameters that govern HLA
object instances and their constituent attributes.
The parameters are discussed in Table~\ref{tab:obj-parameters}.

\begin{table}[h]
  \scriptsize
  \begin{center}
    \begin{tabular}{|p{2in}|l|p{3.25in}|}
    \hline
    parameter name & type & description \\
    \hline \hline
    {\ttfamily THLA.manager.obj\_count}
    ($N_{objs}$)
    & integer
    & the number of object instances to be handled by this federate
    \\
    \hline
    {\ttfamily THLA.manager.objects} & alloc($N_{objs}$)
    & allocate an array with $N_{objs}$ elements
    \\
    \hline
    \hline
    \multicolumn{3}{|c|}{
      \rule[-3mm]{0mm}{7mm}
      let $i \in 0...N_{objs}-1$
    }
    \\
    \hline
    {\ttfamily THLA.manager.objects[$i$].\-FOM\_name}
    & string
    & the HLA class name for object[$i$] as specified in the FOM file
    \\
    \hline
    {\ttfamily THLA.manager.objects[$i$].\-name}
    & string
    & the HLA object instance name for object[$i$]
    \\
    \hline
    {\ttfamily THLA.manager.objects[$i$].\-name\_required}
    & true or false
    & specifies whether the object instance name is required.
    The default is true, which requires an object instance name.
    \\
    \hline
    {\ttfamily THLA.manager.objects[$i$].\-create\_HLA\_instance}
    & true or false
    & specifies whether an HLA object instance is created 
    \\
    \hline
    {\ttfamily THLA.manager.objects[$i$].\-required}
    & true or false
    & specifies whether the object is required for startup of the federation.
    The default is true, which requires the object intance to exist for startup
    of the federation.
    \\
    \hline
    {\ttfamily THLA.manager.objects[$i$].\-blocking\_cyclic\_read}
    & true or false
    & should a cyclic read for this object block waiting for data.
    The default is to not block on cyclic reads.
    \\
    \hline
    {\ttfamily THLA.manager.objects[$i$].\-packing}
    & string
    & the Trick name of a C++ packing object for attributes of object[$i$].
      If no packing is associated with attributes of this object,
      this parameter may be omitted.
    \\
    \hline
    {\ttfamily THLA.manager.objects[$i$].\-ownership}
    & string
    & the Trick name of a C++ ownership handler object for attributes of object[$i$].
      If no ownership transfer is associated with attributes of this object,
      this parameter may be omitted.
    \\
    \hline
    {\ttfamily THLA.manager.objects[$i$].\-deleted}
    & string
    & the Trick name of a C++ object deleted callback object for object [$i$].
      If no object deleted callback is associated with attributes of this object,
      this parameter may be omitted.
    \\
    \hline
    {\ttfamily THLA.manager.objects[$i$].\-attr\_count}
    ($N_{attrs}$)
    & integer
    & the number of attributes associated with object[$i$]
    \\
    \hline
    {\ttfamily THLA.manager.objects[$i$].\-attributes}
    & alloc($N_{attrs}$),
    & allocate an array with $N_{attrs}$ elements
    \\
    \hline
    \hline
    \multicolumn{3}{|c|}{
      \rule[-3mm]{0mm}{7mm}
      let $j \in 0...N_{attrs}-1$
    }
    \\
    \hline
    {\ttfamily THLA.manager.objects[$i$].\-attributes[$j$].FOM\_name}
    & string
    & the HLA name of attribute[$j$] according to the FOM file
    \\
    \hline
    {\ttfamily THLA.manager.objects[$i$].\-attributes[$j$].trick\_name}
    & string
    & the name of the Trick variable with which this attribute is associated.
      When Trick publishes values for this attribute, the values sent out
      are those of the associated Trick variable.
      Similarly,
      when Trick subscribes to this attribute, incoming data are assigned to
      the associated Trick variable.
    \\
    \hline
    {\ttfamily THLA.manager.objects[$i$].\-attributes[$j$].config}
    & {\ttfamily TrickHLA::DataUpdateEnum}
    & Use the enum value {\ttfamily CONFIG\_CYCLIC} to allow this attribte
      to be cyclically sent or received.
    \\
    \hline
    {\ttfamily THLA.manager.objects[$i$].\-attributes[$j$].locally\_owned}
    & true or false
    & is attribute[$j$] initially owned by this federate?
    \\
    \hline
    {\ttfamily THLA.manager.objects[$i$].\-attributes[$j$].publish}
    & true or false
    & is this federate publishing values for attribute[$j$]?
      Federates may only publish values for attributes that are locally owned.
      If this is set to true, \TrickHLA\ will automatically publish values
      for this attribute as specified in the {\ttfamily sim\_object}.
    \\
    \hline
    {\ttfamily THLA.manager.objects[$i$].\-attributes[$j$].subscribe}
    & true or false
    & is this federate subscribing values for attribute[$j$]?
      If this is set to true, \TrickHLA\ will automatically subscribe to values
      for this attribute as specified in the {\ttfamily sim\_object}.
    \\
    \hline
    {\ttfamily THLA.manager.objects[$i$].\-attributes[$j$].preferred\_order}
    & {\ttfamily TrickHLA::TransportationEnum}\footnotemark
    & override the preferred send order of this attribute.
       Use the default enum value {\ttfamily TRANSPORT\_SPECIFIED\_IN\_FOM}
       to use the order specified in FOM.
       Use the enum value  {\ttfamily TRANSPORT\_TIMESTAMP\_ORDER} to override
       the FOM and use a timestamp order.
       Use the enum value  {\ttfamily TRANSPORT\_RECEIVE\_ORDER} to override
       the FOM and use a receive order.
    \\
    \hline
    {\ttfamily THLA.manager.objects[$i$].\-attributes[$j$].rti\_encoding}
    & {\ttfamily TrickHLA::EncodingEnum}\footnotemark
    & the wire format of the values of this attribute.
      This specifies how the values are converted to/from Trick variables.
      Big and little endian formats specify numerical data with the
      associated byte order.
      C strings are null-terminated strings of bytes.
      The Unicode and ASCII string formats are documented in the IEEE Standard
      1516.2-2010, section 4.13.4.
      Opaque data is for non-numeric, non-string data and is also
      documented in 1516.2-2010, section 4.13.6.
    \\
    \hline
    {\ttfamily THLA.manager.objects[$i$].\-attributes[$j$].cycle\_time}
    & integer
    & {\tt default: 1}. Sends cyclic attributes at the specified cycle-time
    provided that the time is an integer multiple of the core job cycle
    time of the send\_cyclic\_data job.
    Ex: A cycle-time of 4 * THLA\_DATA\_CYCLE\_TIME would send this attribute
    every fourth time the send\_cyclic\_data job was called.
    \\
    \hline
    {\ttfamily THLA.manager.objects[$i$].\-attributes[$j$].conditional}
    & {\ttfamily TrickHLAConditional *}
    & {\tt default: NULL}. If overridden, must point to subclass which makes the
      decision if this attribute is to be sent over the wire in each frame.
      Otherwise, this attribute is sent over the wire for each simulation frame.
    \\
    \hline
    \end{tabular}
  \end{center}
  \caption{Object and Attribute-Related Parameters}
  \label{tab:obj-parameters}
\end{table}

\footnotetext{
      The {\ttfamily typedef}, {\ttfamily TrickHLA::TransportationEnum},
      is an enum with the following values:
      {\ttfamily TRANSPORT\_SPECIFIED\_IN\_FOM},
      {\ttfamily TRANSPORT\_TIMESTAMP\_ORDER},
      and {\ttfamily TRANSPORT\_RECEIVE\_ORDER}.
}

\footnotetext{
      The {\ttfamily typedef}, {\ttfamily TrickHLA::EncodingEnum},
      is an enum with the following values:
      {\ttfamily ENCODING\_UNKNOWN}, {\ttfamily ENCODING\_BIG\_ENDIAN}, 
      {\ttfamily ENCODING\_LITTLE\_ENDIAN}, {\ttfamily ENCODING\_LOGICAL\_TIME}, 
      {\ttfamily ENCODING\_C\_STRING}, {\ttfamily ENCODING\_UNICODE\_STRING}, 
      {\ttfamily ENCODING\_ASCII\_STRING}, {\ttfamily ENCODING\_OPAQUE\_DATA}, 
      {\ttfamily ENCODING\_BOOLEAN}, and {\ttfamily ENCODING\_NO\_ENCODING}.
}

\clearpage % force all the tables to be printed before moving on

%%%%%%%%%%%%%%%%%%%%%%%%%%%%%%%%%%%%%%%%%%%%%%%%%%%%%%%%%%%%%%%%%%%%%%%%%
\section{Other HLA Interfaces}

% -----------------------------------------------------------------------
\subsection{Interactions}

% --------------------------------------
\subsubsection{Input Data}

This section describes the input parameters that govern HLA
interactions and their parameters.
The parameters are discussed in Table~\ref{tab:interaction-parameters}.

\begin{table}[hb]
  \scriptsize
  \begin{center}
    \begin{tabular}{|p{2in}|l|p{3.25in}|}
    \hline
    parameter name & type & description \\
    \hline \hline
      {\ttfamily THLA.manager.\-inter\_count}
      ($N_{ints}$)
      & integer
      & the number of interactions to be handled by this federate
      \\
      \hline
      {\ttfamily THLA.manager.\-interactions}
      & alloc($N_{ints}$),
      & allocate an array with $N_{ints}$ elements
      \\
      \hline
      \hline
      \multicolumn{3}{|c|}{
        \rule[-3mm]{0mm}{7mm}
        let $i \in 0...N_{ints}-1$
      }
      \\
      \hline
      {\ttfamily THLA.manager.\-interactions[$i$].FOM\_name}
      & string
      & the HLA name for interaction[$i$] as specified in the FOM file
      \\
      \hline
      {\ttfamily THLA.manager.\-interactions[$i$].publish}
      & true or false
      & can this federate send interaction[$i$] to other federates?
      \\
      \hline
      {\ttfamily THLA.manager.\-interactions[$i$].subscribe}
      & true or false
      & can this federate receive interaction[$i$] from other federates?
      \\
    \hline
    {\ttfamily THLA.manager.\-interactions[$i$].preferred\_order}
    & {\ttfamily TrickHLA::TransportationEnum}
    & override the preferred send order of this interaction.
       Use the default enum value {\ttfamily TRANSPORT\_SPECIFIED\_IN\_FOM}
       to use the order specified in FOM.
       Use the enum value  {\ttfamily TRANSPORT\_TIMESTAMP\_ORDER} to override
       the FOM and use a timestamp order.
       Use the enum value  {\ttfamily TRANSPORT\_RECEIVE\_ORDER} to override
       the FOM and use a receive order.
      \\
      \hline
      {\ttfamily THLA.manager.\-interactions[$i$].handler}
      & string
      & the Trick name of a C++ interaction handler object for
        interaction[$i$].
      \\
      \hline
      {\ttfamily THLA.manager.\-interactions[$i$].param\_count}
      ($N_{params}$)
      & integer
      & the number of parameters associated with interaction[$i$]
      \\
      \hline
      {\ttfamily THLA.manager.\-interactions[$i$].parameters}
      & alloc($N_{params}$)
      & allocate an array with $N_{params}$ elements
      \\
      \hline
      \hline
      \multicolumn{3}{|c|}{
        \rule[-3mm]{0mm}{7mm}
        let $j \in 0...N_{params}-1$
      }
      \\
      \hline
      {\ttfamily
         THLA.manager.\-interactions[$i$].parameters[$j$].\-FOM\_name
      }
      & string
      & the name of parameter[$j$] as specified in the FOM file
      \\
      \hline
      {\ttfamily
         THLA.manager.\-interactions[$i$].parameters[$j$].\-trick\_name
      }
      & string
      & the name of the Trick variable with which this parameter is associated.
      \\
      \hline
      {\ttfamily
         THLA.manager.\-interactions[$i$].parameters[$j$].\-rti\_encoding
      }
      & {\ttfamily TrickHLA::EncodingEnum}
      & the wire format of the values of parameter[$j$].
        The encoding values have the same meaning as the attribute
        encodings discussed in Table~\ref{tab:obj-parameters}.
      \\
      \hline
    \end{tabular}
  \end{center}
  \caption{Interaction-Related Parameters}
  \label{tab:interaction-parameters}
\end{table}

% --------------------------------------
\subsubsection{C++ Handler Classes}

The \TrickHLA\ model includes a base class,
{\ttfamily TrickHLA::InteractionHandler},
which forms the foundation of how developers handle
HLA interactions.
The class implements two {\ttfamily send\_interaction()} methods
(one for receive order and the other for timestamp order transmission
of the interaction),
and it defines a virutal method, {\ttfamily receive\_interaction()},
which must be implemented in subclasses.
(The default implementation of the method does nothing.)

The base class {\ttfamily send\_interaction()} methods may be used
directly to send interactions to remote federates.
You may invoke these methods directly in the simulation
{\ttfamily S\_define} file as scheduled jobs or from internal
simulation code.

In order to receive interactions generated remotely,
you must subclass the base class, overriding the
{\ttfamily receive\_interaction()} method with application-specific logic.
Assuming that the handler(s) have been associated with the appropriate
interactions as discussed in Table~\ref{tab:interaction-parameters},
the \TrickHLA\ infrastructure will invoke the appropriate
{\ttfamily receive\_interaction()} method for each arriving interaction.

In both cases (sending and receiving) the HLA interaction parameter values
are associated with Trick variables as discussed in
Table~\ref{tab:interaction-parameters}.

% -----------------------------------------------------------------------
\subsection{Ownership Transfer}

The \TrickHLA\ model includes an ownership handler class,
{\ttfamily TrickHLA::OwnershipHandler},
which may be used to transfer ownership of specific HLA attributes
of a particular object from one federate to another.
There are two sets of methods that do this.

The {\ttfamily push\_ownership()} methods may be used by a developer to
divest ownership of the specified attributes.
The {\ttfamily pull\_ownership()} methods may be used by a developer to
take over ownership of the specified attributes.
The methods enqueue push/pull requests which are processed by the
jobs in the default \TrickHLA\  {\ttfamily sim\_object}.

Ownership handlers are associated with particular objects through the
input files, as discussed in
Table~\ref{tab:obj-parameters}.

% -----------------------------------------------------------------------
\subsection{Packing and Unpacking}

The \TrickHLA\ model includes a base class,
{\ttfamily TrickHLA::Packing},
which is used for packing HLA data before it is sent to remote federates
and unpacking data just after it has been received.
The class consists of two virtual functions,
{\ttfamily pack()} and {\ttfamily unpack()}, the default implementations
will terminate the simulation,
therefore you must subclass the base class in order to use it,
overriding the two methods.

Packing objects are associated with particular objects through the
input files, as discussed in
Table~\ref{tab:obj-parameters}.

% -----------------------------------------------------------------------
\subsection{Object Deleted}

The \TrickHLA\ model includes a base class,
{\ttfamily TrickHLA::ObjectDeleted},
which is triggered when the object is deleted from the federation.
The class consists of one virtual function,
{\ttfamily deleted()}, the default implementation of which does nothing,
therefore you must subclass the base class in order to use it,
overriding the method.

Object Deleted are associated with particular objects through the
input files, as discussed in
Table~\ref{tab:obj-parameters}.

% -----------------------------------------------------------------------
\subsection{Sending Data Conditionally}

The \TrickHLA\ model includes a class,
{\ttfamily TrickHLA::Conditional},
which must be subclassed by the user's simulation for each attribute that they
wish to send across the wire based upon user-determinted criteria.

The class consists of a method, {\ttfamily should\_send()}, which must be filled
in by the user to provide the aforementioned criteria and return true if it has
been met.

% -----------------------------------------------------------------------
\subsection{Lag Compensation}

\subsubsection{C++ Classes}
The \TrickHLA\ model includes a base class,
{\ttfamily TrickHLA::LagCompensation},
which provides methods that allow federates to compensate for
HLA-induced lags.
The virtual methods,
{\ttfamily send\_lag\_compensation()} and
{\ttfamily receive\_lag\_compensation()}
allow this compensation to be executed by the sender of data or by
the receivers.
The default methods do nothing,
therefore you must subclass the base class in order to use it,
overriding the two methods.

\subsubsection{Input Data}

To associate a lag compensation class with specific HLA objects,
the input data described in Table~\ref{tab:obj-parameters} must be
expanded to include an additional parameter as shown in Table~\ref{tab:lag-parameters}.

\begin{table}[ht]
  \scriptsize
  \begin{center}
    \begin{tabular}{|p{2in}|l|p{3.25in}|}
    \hline
    parameter name & type & description \\
    \hline \hline
      {\ttfamily THLA.manager.\-objects[$i$].\-lag\_comp}
      & string
      & the Trick name of an application-specific lag compensation object.
        This is an instance of a subclass of {\ttfamily TrickHLA::LagCompensation}
        which has been declared in the {\ttfamily S\_define} file.
      \\
      \hline
      {\ttfamily THLA.manager.\-objects[$i$].\-lag\_comp\_type}
      & {\ttfamily TrickHLA::LagCompensation}
      & an enumerated type the specified whether the federate uses
        send-side lag compensation, receive-side lag compensation, or none at all.
        This flag tells the \TrickHLA\ infrastructure whether to invoke
        the {\ttfamily send\_lag\_compensation()} method of the
        specified lag compensation class,
        the {\ttfamily receive\_lag\_compensation()} method, or neither.
      \\
      \hline
    \end{tabular}
  \end{center}
  \caption{Lag Compensation Parameters}
  \label{tab:lag-parameters}
\end{table}

% -----------------------------------------------------------------------
\subsection{Requesting Data Updates}

The \TrickHLA\ model includes a class,
{\ttfamily TrickHLA::Manager},
which manages the HLA data transfers in and out of the user's simulation.

The {\ttfamily TrickHLA::Manager} class includes a method,
{\ttfamily request\_data\_update()},
which may be called by the user's simulation to request a data update for a
specific named object instance.

% -----------------------------------------------------------------------
\subsection{Multiple Verbosity Levels}

The \TrickHLA\ model includes a class, {\ttfamily TrickHLA::DebugHandler},
which is to be used to specify the verbosity level of the \TrickHLA\ software.
It also allows the user the ability to narrow down the specific code modules
that are to emit debug messages.

The settings made to this class need to be done once in the input file, as
described below, and the values are automatically propagated to all subsystems
of the \TrickHLA\ software.

Table~\ref{tab:debug_levels} lists the verbosity levels available to the user,
found in the {\ttfamily TrickHLA::DebugSourceEnum} enumeration, while
table~\ref{tab:code_sections} discusses the code
section control available to the user, found in the {\ttfamily TrickHLA::DebugSourceEnum}
enumeration. Both enumerations are defined in the {\ttfamily TrickHLA/Types.hh}
file.

Setting of the verbosity level is accomplished by setting the class' 
{\ttfamily debug\_level} value, found inside
{\ttfamily THLA.manager.debug\_handler} object, within the simulation's input
file. Note: It is recommended that the user utilize the enumerated values to
specify the desired verbosity level. If the user specifies the
{\ttfamily debug\_level} value via an integer, the value will be range-checked
and any integer value outside of the enumerated values range will recieve a
warning message on the console.

For example, the following command will enable debug level 3 messages to get
emitted on the console:

\begin{verbatim}
// Show or hide the TrickHLA debug messages.
THLA.manager.debug_handler.debug_level = DEBUG_LEVEL_3_TRACE;
\end{verbatim}


Specifying which code sections are to emit debug messages is accomplished by
setting the class' {\ttfamily code\_section} value, found inside
{\ttfamily THLA.manager.debug\_handler} object, within the simulation's input
file. Note: These values are binarily unique so it is recommended that the user
add all desired code sections' enumarated value when specifying them in the
simulation's input file.

For example, the following command will enable debug messages to be emitted only
from {\ttfamily TrickHLA::Manager}, {\ttfamily TrickHLA::Federate} and
{\ttfamily TrickHLA::FedAmb} source code modules:

\begin{verbatim}
THLA.manager.debug_handler.code_section = DEBUG_SOURCE_MANAGER +
                                          DEBUG_SOURCE_FEDERATE +
                                          DEBUG_SOURCE_FED_AMB;
\end{verbatim}


\begin{table}[ht]
  \scriptsize
  \begin{center}
    \begin{tabular}{|p{2in}|l|p{3.25in}|}
    \hline
    enumeration name & value & description \\
    \hline \hline
      {\ttfamily DEBUG\_LEVEL\_NO\_TRACE}
      & 0
      & {\ttfamily default}: Disables all \TrickHLA\ messages and no \TrickHLA\ 
        messages will be displayed on the user's console. Any messages emitted
        by the user's simulation will still be emitted.
      \\
      \hline
      {\ttfamily DEBUG\_LEVEL\_0\_TRACE}
      & 0
      & {\ttfamily default}: Disables all \TrickHLA\ messages and no \TrickHLA\ 
        messages will be displayed on the user's console. Any messages emitted
        by the user's simulation will still be emitted.
      \\
      \hline
      {\ttfamily DEBUG\_LEVEL\_1\_TRACE}
      & 1
      & Adds initialization complete and Time Advance Grant messages.
      \\
      \hline
      {\ttfamily DEBUG\_LEVEL\_2\_TRACE}
      & 2
      & Adds initialization messages as well as the standard complement of execution messages.
      \\
      \hline
      {\ttfamily DEBUG\_LEVEL\_3\_TRACE}
      & 3
      & Adds Ownership Transfer messages.
      \\
      \hline
      {\ttfamily DEBUG\_LEVEL\_4\_TRACE}
      & 4
      & Adds HLA Time Advancement and Freeze job messages.
      \\
      \hline
      {\ttfamily DEBUG\_LEVEL\_5\_TRACE}
      & 5
      & Adds Interaction, InitSyncPts and SyncPts messages.
      \\
      \hline
      {\ttfamily DEBUG\_LEVEL\_6\_TRACE}
      & 6
      & Adds Packing / LagCompensation subclass messages.
      \\
      \hline
      {\ttfamily DEBUG\_LEVEL\_7\_TRACE}
      & 7
      & Adds the names of all Attributes / Parameters sent to other federates.
      \\
      \hline
      {\ttfamily DEBUG\_LEVEL\_8\_TRACE}
      & 8
      & Adds FederateAmbassador and RTI callback messages.
      \\
      \hline
      {\ttfamily DEBUG\_LEVEL\_9\_TRACE}
      & 9
      & Adds Trick Ref-Attributes and RTI Handles (both during initialization).
      \\
      \hline
      {\ttfamily DEBUG\_LEVEL\_10\_TRACE}
      & 10
      & Adds internal state of all Attributes and Parameters.
      \\
      \hline
      {\ttfamily DEBUG\_LEVEL\_11\_TRACE}
      & 11
      & Adds buffer contents of all Attributes and Parameters.
      \\
      \hline
      {\ttfamily DEBUG\_LEVEL\_FULL\_TRACE}
      & 11
      & Adds "all of the above".
      \\
      \hline
    \end{tabular}
  \end{center}
  \caption{TrickHLA::DebugLevelEnum values}
  \label{tab:debug_levels}
\end{table}

\begin{table}[ht]
  \scriptsize
  \begin{center}
    \begin{tabular}{|p{2in}|l|p{3.25in}|}
    \hline
    enumeration name & description \\
    \hline \hline
      {\ttfamily DEBUG\_SOURCE\_FED\_AMB}
      & Adds {\ttfamily TrickHLA::FedAmb} debug messages to the console.
      \\
      \hline
      {\ttfamily DEBUG\_SOURCE\_FEDERATE}
      & Adds {\ttfamily TrickHLA::Federate} debug messages to the console.
      \\
      \hline
      {\ttfamily DEBUG\_SOURCE\_MANAGER}
      & Adds {\ttfamily TrickHLA::Manager} debug messages to the console.
      \\
      \hline
      {\ttfamily DEBUG\_SOURCE\_OBJECT}
      & Adds {\ttfamily TrickHLA::Object} (and subclass) debug messages to the
        console.
      \\
      \hline
      {\ttfamily DEBUG\_SOURCE\_INTERACTION}
      & Adds {\ttfamily TrickHLA::Interaction} (and subclass) debug messages to
        the console.
      \\
      \hline
      {\ttfamily DEBUG\_SOURCE\_ATTRIBUTE}
      & Adds {\ttfamily TrickHLA::Attribute} debug messages to the console.
      \\
      \hline
      {\ttfamily DEBUG\_SOURCE\_PARAMETER}
      & Adds {\ttfamily TrickHLA::Parameter} debug messages to the console.
      \\
      \hline
      {\ttfamily DEBUG\_SOURCE\_SYNCPOINT}
      & Adds {\ttfamily TrickHLA::SyncPnt} debug messages to the console.
      \\
      \hline
      {\ttfamily DEBUG\_SOURCE\_OWNERSHIP}
      & Adds {\ttfamily TrickHLA::OwnershipHandler} debug messages to the console.
      \\
      \hline
      {\ttfamily DEBUG\_SOURCE\_PACKING}
      & Adds {\ttfamily TrickHLAPacking} (and subclass) debug messages to the
        console.
      \\
      \hline
      {\ttfamily DEBUG\_SOURCE\_LAG\_COMPENSATION}
      & Adds {\ttfamily TrickHLA::LagCompensation} (and subclass) debug messages
        to the console.
      \\
      \hline
      {\ttfamily DEBUG\_SOURCE\_ALL\_MODULES}
      & {\ttfamily default}: Adds debug messages from all code modules to the console.
      \\
      \hline
    \end{tabular}
  \end{center}
  \caption{TrickHLA::DebugSourceEnum values}
  \label{tab:code_sections}
\end{table}
