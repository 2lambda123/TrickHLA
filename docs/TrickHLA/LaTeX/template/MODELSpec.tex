%%%%%%%%%%%%%%%%%%%%%%%%%%%%%%%%%%%%%%%%%%%%%%%%%%%%%%%%%%%%%%%%%%%%%%%%%
%
% File: MODELSpec.tex
%
% Purpose: MODEL Product Specification
%
%%%%%%%%%%%%%%%%%%%%%%%%%%%%%%%%%%%%%%%%%%%%%%%%%%%%%%%%%%%%%%%%%%%%%%%%%

\newcommand\documentHistory{
{\bf Author} & {\bf Date} & {\bf Description} \\ \hline \hline
YOUR NAME & DATE & Initial Version \\ \hline
}

\newcommand\DocumentChangeHistory{
{\bf Revised by} & {\bf Date} & {\bf Description} \\ \hline \hline

% This documentation file's change history includes:
% "REVISED BY name(s)" & "revision DATE" & "revision DESCRIPTION"
%  ------------------     -------------     --------------------
                       &                 &                    \\ \hline
}

\documentclass[twoside,11pt,titlepage]{report}

%
% Bring in the amsmath environment
%
\usepackage{amsmath}

%
% Bring in the common page setup
%
\usepackage{trickhlaenv}

%
% Bring in the common math nomenclature
%
\usepackage{trickhlamath}

%
% Bring in the model-specific commands with requirement labels
%
\usepackage[Reqt]{MODEL}

%
% Bring in the graphics environment
%
\usepackage{graphicx}

%
% Bring in listing environment
%
\usepackage{paralist}

%
% Bring in the hyper ref environment
%
\usepackage[colorlinks]{hyperref}
%  keywords for pdfkeywords are separated by commas
\hypersetup{
   pdftitle={\MODEL\ Product Specification},
   pdfauthor={YOUR NAME},
   pdfkeywords={\MODEL, Product Specification},
   pdfsubject={\MODEL\ Product Specification}}

\begin{document}

%%%%%%%%%%%%%%%%%%%%%%%%%%%%%%%%%%%
% Front matter
%%%%%%%%%%%%%%%%%%%%%%%%%%%%%%%%%%%
\pagenumbering{roman}

\docid{DD.mm.03}
\docrev{1.0}
\date{DATE}
\modelname{\MODEL}
\doctype{Product Specification}
\author{YOUR NAME}
\managers{
  Dan E. Dexter \\ Project Manager \\
  Leslie J. Quiocho \\ Common Model Lead \\
  Rodney M. Ondler \\ Software, Robotics, and Simulation Division Chief}
\pdfbookmark{Title Page}{titlepage}
\makeTrickhlaenvTitlepage

\pdfbookmark{Abstract}{abstract}
%%%%%%%%%%%%%%%%%%%%%%%%%%%%%%%%%%%%%%%%%%%%%%%%%%%%%%%%%%%%%%%%%%%%%%%%%
%
% Purpose: Abstract for MODEL
%
% Author: YOUR NAME - DATE
% 
% Modified: 
%  
%
%%%%%%%%%%%%%%%%%%%%%%%%%%%%%%%%%%%%%%%%%%%%%%%%%%%%%%%%%%%%%%%%%%%%%%%%%

\begin{abstract}
This is the abstract of the \MODEL.
\end{abstract}


\pdfbookmark{Contents}{contents}
\tableofcontents
\vfill

\pagebreak

%%%%%%%%%%%%%%%%%%%%%%%%%%%%%%%%%%%
% Main Document Body
%%%%%%%%%%%%%%%%%%%%%%%%%%%%%%%%%%%
\pagenumbering{arabic}

%----------------------------------
\chapter{Introduction}\label{sec:intro}
%----------------------------------

%%%%%%%%%%%%%%%%%%%%%%%%%%%%%%%%%%%%%%%%%%%%%%%%%%%%%%%%%%%%%%%%%%%%%%%%%
%
% Purpose: Introduction for MODEL
%
% Author: YOUR NAME - DATE 
% 
% Modified: 
%  
%
%%%%%%%%%%%%%%%%%%%%%%%%%%%%%%%%%%%%%%%%%%%%%%%%%%%%%%%%%%%%%%%%%%%%%%%%%

\MODEL\ introduction. 


\section{Identification of Document}
This document describes the design of the \MODEL\ developed
for use in the Trick Simulation Environment.
This document adheres to the documentation standards
defined in NASA Software Engineering Requirements Standard \cite{NASA:SWE}.

\section{Scope of Document}
This document provides information on the algorithms used in and
the design of the source code associated with the \MODEL.
This include references to associated texts and the presentation
of various equations for COMPLETE THIS SENTENCE.

\section{Purpose and Objectives of Document}
The purpose of this document is to provide a thorough understanding of the
methods by which the \MODEL\ were defined, programmed, and verified.

\section{Documentation Status and Schedule}
The information in this document is current with the \TrickHLAid\
implementation of the \MODEL. Updates will be kept current with
module changes.

\begin{tabular}{||l|l|l|} \hline
\documentHistory
\end{tabular}

\begin{tabular}{||l|l|l|} \hline
\DocumentChangeHistory
\end{tabular}

\section{Document Contents}
This document is organized into the following sections:

\begin{description}

\item[Chapter \ref{sec:intro}: Introduction] -
Identifies this document, defines the scope and purpose, present status,
and provides a description of each major section.

\item[Chapter \ref{sec:docs}: Related Documentation] -
Lists the related documentation that is applicable to this project.

\item[Chapter \ref{sec:architectural_design}: Architectural Design] -
Presents the top-level concepts behind the \MODEL.

\item[Chapter \ref{sec:math_formulations}: Mathematical Formulations] -
Presents the mathematical formulations implemented by the \MODEL.

\item[Chapter \ref{sec:interface_design}: Interface Design] -
Describes the \MODEL\ data interfaces.

\item[Chapter \ref{sec:functional_design}: Functional Design] -
Presents the detailed design of the model.

\item[Chapter \ref{sec:versions}: Version Description] -
Identifies the configuration-managed items that comprise the \MODEL.

\item[Bibliography] -
Informational references associated with this document.

\item[Appendix \ref{sec:source_code}: Source Code] -
Lists the source code files that comprise the \MODEL.

\item[ADD OTHER APPENDICES IF ANY].

\end{description}

\chapter{Related Documentation}\label{sec:docs}

\section{Parent Documents}
The following documents are parent to this document:

\begin{itemize}
\item{\href{file:\TRICKHLAHOME/docs/TrickHLA.pdf}
           {\em Trick High Level Architecture (\TrickHLA)}}
\cite{trickhlaenv:TrickHLA}

\item{\href{file:MODEL.pdf}
           {\em \MODEL}}
\cite{trickhlaenv:MODEL}
\end{itemize}

\section{Applicable Documents}
The following documents are referenced herein and are directly
applicable to this document:

\begin{itemize}
\item{\href{file:MODELReqt.pdf}
           {\em \MODEL\ Product Requirements}}
\cite{trickhlaenv:MODELReqt}

\item{\href{file:MODELUser.pdf}
           {\em \MODEL\ User Guide}}
\cite{trickhlaenv:MODELUser}

\item{\href{file:MODELIVV.pdf}
           {\em \MODEL\ Inspection, Verification, and Validation}}
\cite{trickhlaenv:MODELIVV}

\item{\em The Trick User's Guide: Trick 2005.0 Release}
\cite{Vetter:TrickUser}

\item{\em Trick Simulation Environment: User Training Materials:
          Trick 2005.0 Release}
\cite{Vetter:TrickUTM}

\item{\em Trick Simulation Environment: Version Description:
          Trick 2005.0 Release}
\cite{Vetter:TrickVD}

\item{\em The Trick Design Document: Trick 2005.0 Release}
\cite{Vetter:TrickDD}

\item{\em NASA Software Engineering Requirements}
\cite{NASA:SWE}
\end{itemize}

\section{Information Documents}
The following documents provide supporting material for understanding the
concepts in this document:

\begin{itemize}
\item{\em SUPPORTING REFERENCE} \cite{SUPPORTINGREFERENCE}
\end{itemize}

\chapter{Architectural Design}\label{sec:architectural_design}
The architectural model for this software is as defined in, and
required by, the Trick Simulation Environment.  The reader is urged
to examine references \cite{Vetter:TrickUTM} and \cite{Vetter:TrickVD}
for a thorough understanding of this programming environment.

\section{Definition of the \MODEL}

\subsection{Purpose and Scope}
The \MODEL\ is a collection of functions that COMPLETE THIS SENTENCE.

This module is not a stand-alone program;
rather, it is designed to be incorporated into the Trick
Simulation Environment, and applies to space vehicles or objects that are
either docked to or undocked from one another.
Primary user(s) of this software are simulation and model developers
who wish to simulate spacecraft operations, including docking and
undocking procedures under nominal and emergency scenarios.

\subsection{Goals and Objectives}
The goal of the \MODEL\ is to COMPLETE THIS PARAGRAPH.


\chapter{Mathematical Formulations}\label{sec:math_formulations}
FILL IN THIS SECTION.
REFERENCE DOCUMENTS AS NEEDED.
REPRODUCE KEY EQUATIONS USED IN THE SOURCE CODE.
DO NOT DESCRIBE THE SOURCE CODE ITSELF YET, THAT COMES LATER.

\chapter{Interface Design}\label{sec:interface_design}

Trick simulations use enumerated types and data structures as the
primary inter-job interface mechanism and for input and output processing.

\section{Data Structure Design}

This section describes the \MODEL\ enumerated types and data structures.

CASE 1: MODEL HAS ONE HEADER FILE
The {\em MODEL.h} file contains COMPLETE THIS SENTENCE.
It contains the following enumerations and data structures:
\begin{description}

\item [ENUM1:] An enumeration of the COMPLETE THIS FRAGMENT.
ADD NECESSARY DETAILS.

\item [ENUMn:] An enumeration of the COMPLETE THIS FRAGMENT.
ADD NECESSARY DETAILS.

\item [STRUCT1:] A data structure containing COMPLETE THIS FRAGMENT.
ADD NECESSARY DETAILS.

\item [STRUCTn:] A data structure containing COMPLETE THIS FRAGMENT.
ADD NECESSARY DETAILS.

\end{description}

% \input{xml/MODEL.h.tex}
% \clearpage

CASE 2: MODEL HAS MULTIPLE HEADER FILE

This section describes the design of the data structures
defined in that file.

The data structures used in conjunction with the
\MODEL\ are defined in the following files:
\begin{description}
\item[MODEL1.h] - BRIEF SYNOPSIS OF MODEL1.h
\item[MODELn.h] - BRIEF SYNOPSIS OF MODELn.h
\end{description}

\subsection{MODEL1.h}

The {\em MODEL1.h} file contains COMPLETE THIS SENTENCE.
It contains the following enumerations and data structures:
\begin{description}

\item [ENUM1:] An enumeration of the COMPLETE THIS FRAGMENT.
ADD NECESSARY DETAILS.

\item [ENUMn:] An enumeration of the COMPLETE THIS FRAGMENT.
ADD NECESSARY DETAILS.

\item [STRUCT1:] A data structure containing COMPLETE THIS FRAGMENT.
ADD NECESSARY DETAILS.

\item [STRUCTn:] A data structure containing COMPLETE THIS FRAGMENT.
ADD NECESSARY DETAILS.

\end{description}

% \input{xml/MODEL1.h.tex}
% \clearpage

\subsection{MODELn.h}

REPEAT ABOVE.


\section{Input Files}

This section describes the default data files used to
initialize the \MODEL\ data structures.

FILL IN THIS SECTION.


\chapter{Functional Design}\label{sec:functional_design}

SAMPLE ONLY - ASSUMES MODEL\_init.c and MODEL.c

This section describes the functions that implement the \MODEL.

These functions are defined in
\begin{description}
\item[\texorpdfstring{\tt MODEL\_init.c}{MODEL\_init.c}] - Initializes model.
\item[\texorpdfstring{\tt MODEL.c}{MODEL.c}] - Updates model.
\end{description}

\section{\texorpdfstring{\tt MODEL\_init.c}{MODEL\_init.c}}

\subsection{Design}

DESCRIBE THE DESIGN OF THIS FILE. REFERENCE ANY MATH MODELS
DESCRIBED IN chapter \ref{sec:architectural_design} THAT
ARE USED IN THIS FUNCTION.
CAPTURE HIGH-LEVEL CONCEPTS THAT ARE NOT IN THE TRICK HEADER.

\subsection{Trick header}

% \input{xml/MODEL_init.tex}

\subsection{Requirements Traceability}

ENUMERATE THE REQUIREMENTS SPECIFIED IN THE REQUIREMENTS
DOCUMENT THAT THIS FUNCTION SATISFIES.


\chapter{Version Description}\label{sec:versions}
This section identifies the versions of the
\MODEL\ described in the current release of the \MODEL\ documentation.

\section{Inventory}
The following items comprise the complete configuration-managed
inventory of the \MODEL:

LIST ALL FILES ASSOCIATED WITH THE MODEL THAT ARE IN THE RAZOR DATABASE.

\section{Change Status}
This is the basic implementation of the software suite.  A change
status is not provided.

\section{Adaptation Data}
This is the first release (basic version) of the \MODEL\ software suite.
No adaptation data is appropriate or is provided, as the user will
define the ab initio data structures at initialization time.

%%%%%%%%%%%%%%%%%%%%%%%%%%%%%%%%%%%%%%%%%%%%%%%%%%%%%%%%%%%%%%%%%%%%%%%%%
% Bibliography
%%%%%%%%%%%%%%%%%%%%%%%%%%%%%%%%%%%%%%%%%%%%%%%%%%%%%%%%%%%%%%%%%%%%%%%%%
\newpage
\pdfbookmark{Bibliography}{bibliography}
\bibliography{trickhlaenv,MODEL}
\bibliographystyle{plain}

%%%%%%%%%%%%%%%%%%%%%%%%%%%%%%%%%%%%%%%%%%%%%%%%%%%%%%%%%%%%%%%%%%%%%%%%%
% Appendices
%%%%%%%%%%%%%%%%%%%%%%%%%%%%%%%%%%%%%%%%%%%%%%%%%%%%%%%%%%%%%%%%%%%%%%%%%
\newpage
\pdfbookmark{Appendix}{appendix}
\appendix


\chapter{Source Code}\label{sec:source_code}
INCLUDE ALL .H, .D, and .C files ASSOCIATED WITH THE MODEL.
PRECEDE EACH UNDERSCORE IN A SUBSECTION LINE WITH A BACKSLASH.

\section{C Header Files}

\subsection{\texorpdfstring{\tt MODEL.h}{MODEL.h}}
% \input{include/MODEL.h.tex}
\pagebreak

\section{Data Files}
\subsection{\texorpdfstring{\tt MODEL.d}{MODEL.d}}
% \input{data/MODEL.d.tex}
\pagebreak

\section{C Source Files}
\subsection{\texorpdfstring{\tt MODEL.c}{MODEL.c}}
% %%%%%%%%%%%%%%%%%%%%%%%%%%%%%%%%%%%%%%%%%%%%%%%%%%%%%%%%%%%%%%%%%%%%%%%%%
%
% File: MODEL.tex
%
% Purpose: Top level document for MODEL
%
%%%%%%%%%%%%%%%%%%%%%%%%%%%%%%%%%%%%%%%%%%%%%%%%%%%%%%%%%%%%%%%%%%%%%%%%%

\newcommand\documentHistory{
{\bf Author} & {\bf Date} & {\bf Description} \\ \hline \hline
YOUR NAME & DATE & Initial Version \\ \hline
}

\newcommand\DocumentChangeHistory{
{\bf Revised by} & {\bf Date} & {\bf Description} \\ \hline \hline

% This documentation file's change history includes:
% "REVISED BY name(s)" & "revision DATE" & "revision DESCRIPTION"
%  ------------------     -------------     --------------------
                       &                 &                    \\ \hline
}

\documentclass[twoside,11pt,titlepage]{report}

%
% Bring in the common page setup
%
\usepackage{trickhlaenv}

%
% Bring in the model-specific commands
%
\usepackage{MODEL}

%
% Bring in the graphics environment
%
\usepackage{graphicx}

%
% Bring in the hyper ref environment
%
\usepackage[colorlinks]{hyperref}
%  keywords for pdfkeywords are separated by commas
\hypersetup{
   pdftitle={\MODEL},
   pdfauthor={YOUR NAME},
   pdfkeywords={\MODEL},
   pdfsubject={\MODEL}}

\begin{document}

%%%%%%%%%%%%%%%%%%%%%%%%%%%%%%%%%%%
% Front matter
%%%%%%%%%%%%%%%%%%%%%%%%%%%%%%%%%%%
\pagenumbering{roman}

\docid{DD.mm.01}
\docrev{1.0}
\date{DATE}
\modelname{\MODEL}
\doctype{}
\author{YOUR NAME}
\managers{
  Edwin Z. Crues \\ Project Manager \\
  Michael T. Red \\ Simulation and Graphics Branch Chief (ER7) \\
  Robert O. Ambrose \\ Software, Robotics, and Simulation Division Chief}
\pdfbookmark{Title Page}{titlepage}
\makeTrickhlaenvTitlepage

\pdfbookmark{Abstract}{abstract}
%%%%%%%%%%%%%%%%%%%%%%%%%%%%%%%%%%%%%%%%%%%%%%%%%%%%%%%%%%%%%%%%%%%%%%%%%
%
% Purpose: Abstract for MODEL
%
% Author: YOUR NAME - DATE
% 
% Modified: 
%  
%
%%%%%%%%%%%%%%%%%%%%%%%%%%%%%%%%%%%%%%%%%%%%%%%%%%%%%%%%%%%%%%%%%%%%%%%%%

\begin{abstract}
This is the abstract of the \MODEL.
\end{abstract}


\pdfbookmark{Contents}{contents}
\tableofcontents
\vfill

\pagebreak

%%%%%%%%%%%%%%%%%%%%%%%%%%%%%%%%%%%
% Main Document Body
%%%%%%%%%%%%%%%%%%%%%%%%%%%%%%%%%%%
\pagenumbering{arabic}

%----------------------------------
\chapter{Introduction}
%----------------------------------
%%%%%%%%%%%%%%%%%%%%%%%%%%%%%%%%%%%%%%%%%%%%%%%%%%%%%%%%%%%%%%%%%%%%%%%%%
%
% Purpose: Introduction for MODEL
%
% Author: YOUR NAME - DATE 
% 
% Modified: 
%  
%
%%%%%%%%%%%%%%%%%%%%%%%%%%%%%%%%%%%%%%%%%%%%%%%%%%%%%%%%%%%%%%%%%%%%%%%%%

\MODEL\ introduction. 


\section{Identification of Document}
This document describes the \MODEL\ developed
for use in the Trick Simulation Environment.
This document adheres to the documentation standards
defined in NASA Software Engineering Requirements Standard \cite{NASA:SWE}.

\section{Scope of Document}
This document provides information on the
\begin{itemize}
\item the requirements for,
\item the algorithms used in and design of,
\item verification and validation of, and
\item the use of
\end{itemize}
the \MODEL. This included references to associated texts and the
presentation of various equations for COMPLETE THIS SENTENCE.

\section{Purpose and Objectives of Document}
The purpose of this document is to provide a thorough understanding of the
methods by which the \MODEL\ were defined, programmed, and verified.

\section{Documentation Status and Schedule}
The information in this document is current with the \TrickHLAid\ implementation
of the \MODEL\ modules. Updates will be kept current with module changes.

\begin{tabular}{||l|l|l|} \hline
\documentHistory
\end{tabular}

\begin{tabular}{||l|l|l|} \hline
\DocumentChangeHistory
\end{tabular}

\section{Documentation Organization}
This document is formatted in accordance with the
NASA Software Engineering Requirements Standard \cite{NASA:SWE}
and is organized into the following chapters:

\begin{description}

\item[Chapter 1: Introduction] -
Identifies this document, defines the scope and purpose, present status,
and provides a description of each major section.

\item[Chapter 2: Related Documentation] -
Lists the related documentation that is applicable to this project.

\item[Chapter 3: Product Requirements] -
Describes requirements for the \MODEL.

\item[Chapter 4: Product Specification] -
Describes the underlying theory, architecture, and design of
the \MODEL\ in detail.

\item[Chapter 5: User Guide] -
Describes how to use the \MODEL\ in a Trick simulation.

\item[Chapter 6: Inspection, Verification, and Validation] -
Contains \MODEL\ inspection, verification, and validation
procedures and results.

\end{description}

\chapter{Related Documentation}

\section{Parent Documents}

The following document is parent to this document:
\begin{itemize}
\item{\href{file:\TRICKHLAHOME/docs/TrickHLA.pdf}
           {\em Trick High Level Architecture (\TrickHLA)}}
\cite{trickhlaenv:TrickHLA}
\end{itemize}

\section{Applicable Documents}
The following documents are referenced herein and are directly
applicable to this document:

\begin{itemize}
\item{\href{file:MODELReqt.pdf}
           {\em \MODEL\ Product Requirements}}
\cite{trickhlaenv:MODELReqt}

\item{\href{file:MODELSpec.pdf}
           {\em \MODEL\ Product Specification}}
\cite{trickhlaenv:MODELSpec}

\item{\href{file:MODELUser.pdf}
           {\em \MODEL\ User Guide}}
\cite{trickhlaenv:MODELUser}

\item{\href{file:MODELIVV.pdf}
           {\em \MODEL\ Inspection, Verification, and Validation}}
\cite{trickhlaenv:MODELIVV}

\item{\em The Trick User's Guide: Trick 2005.0 Release}
\cite{Vetter:TrickUser}

\item{\em Trick Simulation Environment: User Training Materials:
          Trick 2005.0 Release}
\cite{Vetter:TrickUTM}

\item{\em Trick Simulation Environment: Version Description:
          Trick 2005.0 Release}
\cite{Vetter:TrickVD}

\item{\em The Trick Design Document: Trick 2005.0 Release}
\cite{Vetter:TrickDD}

\item{\em NASA Software Engineering Requirements}
\cite{NASA:SWE}

\item{\em OTHER REFERENCE} \\cite{OTHERREFERENCE}HERE
\end{itemize}

\section{Information Documents}
The following documents provide supporting material for understanding the
concepts in this document:

\begin{itemize}
\item{\em SUPPORTING REFERENCE} \\cite{SUPPORTINGREFERENCE}HERE
\end{itemize}

See the bibliography for the details associated with these references.


\chapter{Product Requirements}

This section of the documentation has been rolled out into a
\href{file:MODELReqt.pdf}{separate document}
\cite{trickhlaenv:MODELReqt}.

\chapter{Product Specification}

This section of the documentation has been rolled out into a
\href{file:MODELSpec.pdf}{separate document}
\cite{trickhlaenv:MODELSpec}.

\chapter{User Guide}

This section of the documentation has been rolled out into a
\href{file:MODELUser.pdf}{separate document}
\cite{trickhlaenv:MODELUser}.

\chapter{Inspection, Verification, and Validation}

This section of the documentation has been rolled out into a
\href{file:MODELIVV.pdf}{separate document}
\cite{trickhlaenv:MODELIVV}.

%%%%%%%%%%%%%%%%%%%%%%%%%%%%%%%%%%%%%%%%%%%%%%%%%%%%%%%%%%%%%%%%%%%%%%%%%
% Bibliography
%%%%%%%%%%%%%%%%%%%%%%%%%%%%%%%%%%%%%%%%%%%%%%%%%%%%%%%%%%%%%%%%%%%%%%%%%
\newpage
\pdfbookmark{Bibliography}{bibliography}
\bibliography{trickhlaenv,MODEL}
\bibliographystyle{plain}

%\pagebreak
%\appendix

\end{document}

\pagebreak

\end{document}
