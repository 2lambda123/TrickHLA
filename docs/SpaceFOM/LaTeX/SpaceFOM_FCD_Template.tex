%%%%%%%%%%%%%%%%%%%%%%%%%%%%%%%%%%%%%%%%%%%%%%%%%%%%%%%%%%%%%%%%%%%%%%%%%%%%%%
%%% TrickHLA SpaceFOM FCD Template
%%%%%%%%%%%%%%%%%%%%%%%%%%%%%%%%%%%%%%%%%%%%%%%%%%%%%%%%%%%%%%%%%%%%%%%%%%%%%%
%%%
%%% Purpose:
%%%   This document provides a LaTeX based implementation of the SISO
%%%   SpaceFOM Federate Compliance Document (FCD).  This is a template
%%%   document that is intended to provide a starting point for anyone that
%%%   needs to create SpaceFOM-compliant FCD.
%%%
%%%   The document contains editorial notes, in red.  These notes can be
%%%   revealed or hidden using the LaTeX 'multiaudience' package.  This
%%%   document also contains example place holder text, marked with the
%%%   \example{} text entries.  These examples are intended to be replaced
%%%   with normal and appropriate text entries.
%%%
%%% Revision History:
%%%   January 2021: Edwin Z. Crues: Initial version based on SISO templates.
%%%
%%%%%%%%%%%%%%%%%%%%%%%%%%%%%%%%%%%%%%%%%%%%%%%%%%%%%%%%%%%%%%%%%%%%%%%%%%%%%%

%%%%%%%%%%%%%%%%%%%%%%%%%%%%%%
% Primary document style
\documentclass[11pt,english,letterpaper]{article}
\setlength{\parskip}{\baselineskip}
\setlength{\parindent}{0pt}
%%%%%%%%%%%%%%%%%%%%%%%%%%%%%%

%%%%%%%%%%%%%%%%%%%%%%%%%%%%%%
% Required packages
\usepackage[english]{babel}
%\usepackage{helvet}
\usepackage{textcomp}
\usepackage{enumitem}
\usepackage{xcolor}
\usepackage{multiaudience}
\usepackage{fancyhdr} % needed for header and footer
\usepackage{lastpage} % needed for page numbering 1 of 2, 2 of 2, etc...
\usepackage[colorlinks]{hyperref}
\hypersetup{
   colorlinks=true,
   linkcolor=blue,
   filecolor=green,
   citecolor=blue,
   urlcolor=red,
   plainpages=false,
   pdfpagelabels=true
}
\usepackage{biblatex}
\usepackage{tabularx} % Better tables.
%%%%%%%%%%%%%%%%%%%%%%%%%%%%%%

%%%%%%%%%%%%%%%%%%%%%%%%%%%%%%
% Page spacing options
\topmargin 0in 
\headheight 0.25in
\headsep 12pt  
\textheight 9in 
\textwidth 6.5in
\oddsidemargin 0in
\evensidemargin 0in
%%%%%%%%%%%%%%%%%%%%%%%%%%%%%%

%%%%%%%%%%%%%%%%%%%%%%%%%%%%%%%%%%%%%%%%%%%%%%%%%%%%%%%%%%%%%%%%%%%%%%%%%%%%%%
% End of preamble
%%%%%%%%%%%%%%%%%%%%%%%%%%%%%%%%%%%%%%%%%%%%%%%%%%%%%%%%%%%%%%%%%%%%%%%%%%%%%%
%%%%%%%%%%%%%%%%%%%%%%%%%%%%%%%%%%%%%%%%%%%%%%%%%%%%%%%%%%%%%%%%%%%%%%%%%%%%%%

\SetNewAudience{editorial}
\DefCurrentAudience{editorial} % Uncomment this to see editorial text.

\newcommand{\example}[1]{{\textcolor{blue}{\textit{#1}}}}

\addbibresource{IEEE1516.bib}
\addbibresource{SpaceFOM_FCD_Template.bib}

%%%%%%%%%%%%%%%%%%%%%%%%%%%%%%%%%%%%%%%%%%%%%%%%%%%%%%%%%%%%%%%%%%%%%%%%%%%%%%
%%%%%%%%%%%%%%%%%%%%%%%%%%%%%%%%%%%%%%%%%%%%%%%%%%%%%%%%%%%%%%%%%%%%%%%%%%%%%%
% Starting Document Content
%%%%%%%%%%%%%%%%%%%%%%%%%%%%%%%%%%%%%%%%%%%%%%%%%%%%%%%%%%%%%%%%%%%%%%%%%%%%%%
\begin{document}

\begin{center}
\begin{LARGE}
{\bfseries Space Reference Federation Object Model\\
(SpaceFOM)\\
Federate Compliance Document (FCD)}\\
\vspace{10pt}
\end{LARGE}
{\normalsize for the}\\
\vspace{10pt}
\begin{LARGE}
{\bfseries \example{\textlangle{}Federate Name\textrangle{}}}
\end{LARGE}
\end{center}

\begin{shownto}{editorial}
{\color{red} The information that appears in red text is intended
to be editorial and directive content.  It is not intended to appear in the
finished document.  This text can be hidden by commenting out or deleting the

\texttt{\textbackslash DefCurrentAudience\{editorial\}}

line in this document's \texttt{*.tex} file.  The \example{blue italic text}
represents example text and is intended to be replaced with the Federate
specific text.  When complete, the document should not have any red
or {\color{blue}blue} text, other than hyprelinked text.

The Space Reference Federation Object Model (SpaceFOM) Federate Compliance
Declaration (FCD) is a document that provides specific configuration data
necessary to achieve interoperability based on the SpaceFOM. Several rules
in the SpaceFOM put requirements on what data need to be recorded in the FCD.
In general, The FCD describes which capabilities a federate has and which roles
it can play in a SpaceFOM-compliant federation execution. This template
establishes the standard format and content so that all SpaceFOM FCD products
contain the same basic information and have the same basic look.}
\end{shownto}


%%%%%%%%%%%%%%%%%%%%%%%%%%%%%%%%%%%%%%%%%%%%%%%%%%%%%%%%%%%%%%%%%%%%%%%%%%%%%%
\section*{Purpose}

\example{ This section of the FCD template will provide the general purpose and
description of this specific SpaceFOM-compliant federate. This should include
intended scenarios and other information that describes the nature of the
federate's capabilities and compliance as a SpaceFOM-compliant federate.
Federate providers should provide a federate compliance declarations to
facilitate the assessment of the suitability of a federate in a specific
federation execution. \cite{IEEE1516:FRAMEWORK,IEEE1516:API,IEEE1516:OMT}}


%%%%%%%%%%%%%%%%%%%%%%%%%%%%%%%%%%%%%%%%%%%%%%%%%%%%%%%%%%%%%%%%%%%%%%%%%%%%%%
\section*{Identification}

\begin{shownto}{editorial}
{\color{red} This section of the FCD template provides the general identifying
information associates with this federate.

General name identifying this federate, this should match the
\textlangle{}Federate Name\textrangle{} in the title above but not necessarily
the name used when the federate joins an HLA federation execution (see ``HLA
Federation Execution Join Name'' below).}
\end{shownto}

\textbf{Name: } \underline{\example{This Federate's Name}}

\begin{shownto}{editorial}
{\color{red} Specify the federate version identification.}
\end{shownto}

\textbf{Version: } \underline{\example{Version Tag}}

\begin{shownto}{editorial}
{\color{red} Information pertaining to the principal point of contact for
this FCD.}
\end{shownto}

\textbf{Point of Contact: }

\hspace{0.25in}
\begin{tabularx}{\textwidth}{lX}
Name:    & \underline{\example{POC Name}} \\
Phone:   & \underline{\example{POC Phone Number}} \\
Email:   & \underline{\example{POC Email Address}} \\
Address: & \underline{\example{POC Phydical Address}}
\end{tabularx}

\begin{shownto}{editorial}
{\color{red} The HLA name used by this federate when joining an HLA federation
execution, not necessarily the identification name from above. Specify the
name here if it is fixed or indicate the means for setting if it can be
configured at runtime.}
\end{shownto}

\textbf{HLA Federation Execution Join Name: } \underline{\example{Federate Instance Join Name}}


%%%%%%%%%%%%%%%%%%%%%%%%%%%%%%%%%%%%%%%%%%%%%%%%%%%%%%%%%%%%%%%%%%%%%%%%%%%%%%
\section*{SpaceFOM Federate Roles Supported}

\begin{shownto}{editorial}
{\color{red} This section of the FCD template provides information on the
SpaceFOM roles that this federate can support.

Statement that this federate can fulfill the role of the Master Federate in a
SpaceFOM-compliant federation execution. This is a yes or no question.}
\end{shownto}

\textbf{Can act as Master Federate: } \underline{\example{(yes/no)}}

\begin{shownto}{editorial}
{\color{red} Statement that this federate can fulfill the role of the Pacing
Federate in a SpaceFOM-compliant federation execution. This is a yes or no
question.}
\end{shownto}

\textbf{Can act as Pacing Federate: } \underline{\example{(yes/no)}}

\begin{shownto}{editorial}
{\color{red} Statement that this federate can fulfill the role of the Root
Reference Frame Publisher (RRFP) federate in a SpaceFOM-compliant federation
execution. This is a yes or no question.}
\end{shownto}

\textbf{Can act as Root Reference Frame Publisher: } \underline{\example{(yes/no)}}


%%%%%%%%%%%%%%%%%%%%%%%%%%%%%%%%%%%%%%%%%%%%%%%%%%%%%%%%%%%%%%%%%%%%%%%%%%%%%%
\section*{Time Management}

\begin{shownto}{editorial}
{\color{red} This section of the FCD template provides the general time
management information associates with this federate.

Specify the earliest and latest valid operating Simulation Scenario Time (SST)
dates and times for this federate. This can be given as a calendar date and time
but will ultimately have to be converted into the Terrestrial Time (TT) scale in
Truncated Julian Date (TJD) format.}
\end{shownto}

\textbf{Valid Operating Time Frame: }

\hspace{0.25in}
\begin{tabular}{ll}
Earliest:& \underline{\example{Earliest Date}} (TT scale in TJD format)\\
Latest:  & \underline{\example{Latest Date}} (TT scale in TJD format)
\end{tabular}

\begin{shownto}{editorial}
{\color{red}Specify the minimum, nominal, and maximum HLA Logical Time (HLT)
step in microseconds supported by this federate. These time step capabilities
will inform the Least Common Time Step (LCTS) calculation for any federation
execution that this federate joins.}
\end{shownto}

\textbf{Time Step Support: }

\hspace{0.25in}
\begin{tabular}{ll}
Minimum: & \underline{\example{Smallest time step}} (microseconds)\\
Nominal: & \underline{\example{Normal time step}} (microseconds)\\
Maximum: & \underline{\example{Largest time step}} (microseconds)
\end{tabular}

\begin{shownto}{editorial}
{\color{red} Indication that this federate supports being an Early Joiner
federate. This is a yes or no question.}
\end{shownto}

\textbf{Supports Early Joining: } \underline{ \example{(yes/no)}}

\begin{shownto}{editorial}
{\color{red} Indication that this federate supports being a Late Joiner
federate. This is a yes or no question.}
\end{shownto}

\textbf{Supports Late Joining: } \underline{ \example{(yes/no)}}

\begin{shownto}{editorial}
{\color{red} Specify if this federate can or should be a time regulating federate.}
\end{shownto}

\textbf{Time Regulating: } \underline{ \example{(required/optional/no)}}

\begin{shownto}{editorial}
{\color{red} Specify if this federate can or should be a time constrained federate.}
\end{shownto}

\textbf{Time Constrained: } \underline{ \example{(required/optional/no)}}

\begin{shownto}{editorial}
{\color{red} Identify the supported time management type for this federate.
These are yes or no questions.}
\end{shownto}

\textbf{Supported Time Management Types: }

\hspace{0.25in}
\begin{tabular}{ll}
No Pacing:                                & \underline{\example{(yes/no)}} \\
Scaled Pacing:                            & \underline{\example{(yes/no)}} \\
Real-time Pacing with Unlimited Overruns: & \underline{\example{(yes/no)}} \\
Real-time Pacing with Limited Overruns:   & \underline{\example{(yes/no)}} \\
Strict/Conservative Real-time Pacing:     & \underline{\example{(yes/no)}} \\
\end{tabular}

\begin{shownto}{editorial}
{\color{red} If any of the real-time pacing options are supported, then
include a section that describes limitations and how those overruns are
handled.}
\end{shownto}

\textbf{Overrun handling: } \underline{\example{Description of how overruns
are handled.}}

\begin{shownto}{editorial}
{\color{red} Indication that this federate requires support for Central
Timing Equipment (CTE) to control its time advance. This is a yes or no
question.}
\end{shownto}

\textbf{Requires CTE: } \underline{\example{(yes/no)}}

\begin{shownto}{editorial}
{\color{red} References to any document(s) that describes the implementation
and configuration details for any CTE. Add additional document references as
necessary. Just mark (N/A) if CTE is not required.}
\end{shownto}

\textbf{CTE specification document(s): } \underline{\example{(yes/no)}}
\begin{enumerate}
\item \example{CTE reference document 1.}
\item \example{CTE reference document 2.}
\end{enumerate}


%%%%%%%%%%%%%%%%%%%%%%%%%%%%%%%%%%%%%%%%%%%%%%%%%%%%%%%%%%%%%%%%%%%%%%%%%%%%%%
\section*{Reference Frames}

\begin{shownto}{editorial}
{\color{red} This section of the FCD template provides the names and
descriptions of the principal reference frames published by or required by
this federate.

If this federate can publish a root reference frame, then this is the name and
brief description of the reference frame that represents the common base (root)
reference frame for a SpaceFOM-compliant reference frame tree. If this federate
cannot publish a root reference frame, then this entry should be omitted.}
\end{shownto}

\textbf{Root Reference Frame: }

\begin{tabularx}{\textwidth}{lX}
Name & Description \\
\example{Root Frame Name} &
\example{A short description of the root reference frame.  If not apparent from
the name and short description, a reference should be provided.\cite{vallado2001}}
\end{tabularx}

\begin{shownto}{editorial}
{\color{red} The name, parent, and brief description of any reference frames
published by this federate. Add additional lines as necessary.}
\end{shownto}

\textbf{Published Reference Frames: }

\begin{tabularx}{\textwidth}{|l|l|X|} \hline
Name & Parent & Description \\ \hline
\example{Frame name} &
\example{Name of parent frame} &
\example{Description of this federate's reference frame.} \\ \hline
\example{etc.} &
\example{Name of parent frame} &
\example{etc.} \\ \hline
\end{tabularx}

\begin{shownto}{editorial}
{\color{red} The name, parent, and brief description of any reference frames
required by this federate. Add additional lines as necessary.}
\end{shownto}

\textbf{Required Reference Frames: }

\begin{tabularx}{\textwidth}{|l|X|} \hline
Name & Description \\ \hline
\example{Root frame name} &
\example{Description of the frame dependency.} \\ \hline
\example{EarthCenteredInertial} &
\example{All \texttt{DynamicalEntity} states are expressed in this frame.} \\ \hline
\example{etc.} &
\example{etc.} \\ \hline
\end{tabularx}


%%%%%%%%%%%%%%%%%%%%%%%%%%%%%%%%%%%%%%%%%%%%%%%%%%%%%%%%%%%%%%%%%%%%%%%%%%%%%%
\section*{Object Management}

\begin{shownto}{editorial}
{\color{red} This section of the FCD template provides the general object
management information associated with the federate. The information in this
section will inform the overall object management strategy for a federation
execution in which this federate participates.

List the type strings associated with any PhysicalEntity object's ``type''
attribute used in this federate. List both the string (tag) values and a
description of each tag.}
\end{shownto}

\textbf{Physical Entity Object Type Strings: }

\begin{tabularx}{\textwidth}{|l|X|} \hline
Type String (Tag) & Description \\ \hline
\example{Tag 1} & \example{The description of the tag.} \\ \hline
\example{Tag 2} & \example{The description of the tag.} \\ \hline
\example{etc.} & \\ \hline
\end{tabularx}

\begin{shownto}{editorial}
{\color{red} List the status strings associated with any PhysicalEntity
object's ``status'' attribute used in this federate. List both the string (tag)
values and a description of each tag.}
\end{shownto}

\textbf{Physical Entity Object Status Strings: }

\begin{tabularx}{\textwidth}{|l|X|} \hline
Status String (Tag) & Description \\ \hline
\example{String 1} & \example{The description of the string.} \\ \hline
\example{String 2} & \example{The description of the string.} \\ \hline
\example{etc.} &  \\ \hline
\end{tabularx}

\begin{shownto}{editorial}
{\color{red} A brief description of the canonical naming convention used
to distinguish PhysicalInterface object instances from one another. Add
additional lines as necessary.}
\end{shownto}

\textbf{Physical Interface Instance Naming Convention: }

\example{Provide a detailed explanation of the \texttt{PhysicalInterface}
instance naming convention used to uniquely identify the interfaces to
\texttt{PhysicalEntities}.}

\begin{shownto}{editorial}
{\color{red} List the name and brief description of all \texttt{PhysicalInterface}
instances. Add additional lines as necessary.}
\end{shownto}

\textbf{Physical Interface Instances: }

\begin{tabularx}{\textwidth}{|l|X|} \hline
Interface Instance Name & Description \\ \hline
\example{Instance Name 1} & 
\example{The description of the string.} \\ \hline
\example{Instance Name 2} & 
\example{The description of the string.} \\ \hline
\example{etc.} & \\ \hline
\end{tabularx}

\begin{shownto}{editorial}
{\color{red} List the name, type, and brief description of any published
object instances. Add additional lines as necessary.}
\end{shownto}

\textbf{Published Object Instances: }

\begin{tabularx}{\textwidth}{|l|l|X|} \hline
Name & Type & Description \\ \hline
\example{Instance Name 1} & \example{\texttt{ObjectType}} &
\example{The description of the string.} \\ \hline
\example{Instance Name 2} & \example{\texttt{ObjectType}} &
\example{The description of the string.} \\ \hline
\example{etc.} & \\ \hline
\end{tabularx}




\begin{shownto}{editorial}
{\color{red} Editorial text.}
\end{shownto}

\begin{shownto}{editorial}
{\color{red} List the name, type, and brief description of any required
object instances. Add additional lines as necessary.}
\end{shownto}

\textbf{Required Object Instances: }

\begin{tabularx}{\textwidth}{|l|l|X|} \hline
Name & Type & Description \\ \hline
\example{Instance Name 1} & \example{\texttt{ObjectType}} &
\example{The description of the string.} \\ \hline
\example{Instance Name 2} & \example{\texttt{ObjectType}} &
\example{The description of the string.} \\ \hline
\example{etc.} & \\ \hline
\end{tabularx}

\begin{shownto}{editorial}
{\color{red}List the name and description of any additional FOM modules need by
this federate. Add additional lines as necessary.}
\end{shownto}

\textbf{Additional FOM Modules: }

\begin{tabularx}{\textwidth}{|l|X|} \hline
FOM Module Name & Description \\ \hline
\example{Module Name 1} & \example{The description of the FOM module.} \\ \hline
\example{Module Name 2} & \example{The description of the FOM module.} \\ \hline
\example{etc.} &  \\ \hline
\end{tabularx}


%%%%%%%%%%%%%%%%%%%%%%%%%%%%%%%%%%%%%%%%%%%%%%%%%%%%%%%%%%%%%%%%%%%%%%%%%%%%%%
\section*{Initialization}

\begin{shownto}{editorial}
{\color{red} This section of the FCD template provides the information
associates with the initialization policy and approach use by this federate. It
specifically focuses on the details of its multiphase initialization process.
The information in this section will inform the overall MPI strategy for a
federation execution in which this federate participates.

Indication for the use of multiphase initialization (MPI). This is a yes or no
question.}
\end{shownto}

\textbf{MPI Used: } \underline{\example{(yes/no)}}

\begin{shownto}{editorial}
{\color{red} The MPI specification can consist of an inline description of the
MPI approach. Alternately, list references to any document(s) that describes the
implementation and configuration details for any MPI used in the startup of the
federation execution. Add additional document references as necessary. Just mark
(N/A) if no MPI is used.}
\end{shownto}

\textbf{MPI Specification: }
\begin{enumerate}
\item \example{MPI reference document 1.}
\item \example{MPI reference document 2.}
\end{enumerate}


%%%%%%%%%%%%%%%%%%%%%%%%%%%%%%%%%%%%%%%%%%%%%%%%%%%%%%%%%%%%%%%%%%%%%%%%%%%%%%
\section*{Additional Technical Information}

\begin{shownto}{editorial}
{\color{red} This section of the FCD template provides any additional technical
information needed by this federate. This section may be marked (N/A) or omitted
if there is no additional technical information.

List or describe any additional data sources and/or databases required to
support this federate. Add additional lines as necessary. Just mark (N/A) or
omit if no additional data sources are needed.}
\end{shownto}

\textbf{Additional Data and/or Databases: }

\begin{tabularx}{\textwidth}{|l|X|} \hline
\textbf{Data Source} & \textbf{Data Description} \\ \hline
\example{Source 1} & 
\example{The description of the data.} \\ \hline
\example{Source 2} & 
\example{The description of the data.} \\ \hline
\example{etc.} &  \\ \hline
\end{tabularx}

\begin{shownto}{editorial}
{\color{red} References to any additional technical document. Add additional
document references as necessary. Just mark (N/A) or omit if none.}
\end{shownto}

\textbf{Additional Technical Documents: }
\begin{enumerate}
\item \example{Technical reference document 1.}
\item \example{Technical reference document 2.}
\end{enumerate}


%%%%%%%%%%%%%%%%%%%%%%%%%%%%%%%%%%%%%%%%%%%%%%%%%%%%%%%%%%%%%%%%%%%%%%%%%%%%%%
\section*{Compliance Statement}

\begin{shownto}{editorial}
{\color{red} This is the general acknowledgement that this federate complies
with the Space Reference FOM standard. This is a yes or no question.}
\end{shownto}

This federate fulfills all relevant requirements in the SpaceFOM version 1.0: \underline{\example{(yes/no)}}

\begin{shownto}{editorial}
{\color{red} If the answer to the compliance question above is “no”, then an
explanatory note can be provided here.}
\end{shownto}


\begin{shownto}{editorial}
{\color{red} The next section is automatically generated by \LaTeX using the
BibLaTeX pacakge and Biber program.  A bibliography will be generated from
\LaTeX citations that occur in the document.}
\end{shownto}

%%%%%%%%%%%%%%%%%%%%%%%%%%%%%%%%%%%%%%%%%%%%%%%%%%%%%%%%%%%%%%%%%%%%%%%%%%%%%%
\printbibliography

\end{document}
%%%%%%%%%%%%%%%%%%%%%%%%%%%%%%%%%%%%%%%%%%%%%%%%%%%%%%%%%%%%%%%%%%%%%%%%%%%%%%
% End Document Content
%%%%%%%%%%%%%%%%%%%%%%%%%%%%%%%%%%%%%%%%%%%%%%%%%%%%%%%%%%%%%%%%%%%%%%%%%%%%%%
%%%%%%%%%%%%%%%%%%%%%%%%%%%%%%%%%%%%%%%%%%%%%%%%%%%%%%%%%%%%%%%%%%%%%%%%%%%%%%
